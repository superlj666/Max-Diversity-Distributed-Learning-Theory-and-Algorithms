\documentclass{article}

% if you need to pass options to natbib, use, e.g.:
% \PassOptionsToPackage{numbers, compress}{natbib}
% before loading nips_2018

% ready for submission
\usepackage{nips_2018}
\usepackage{algorithm}
\usepackage{algorithmic}
% to compile a preprint version, e.g., for submission to arXiv, add
% add the [preprint] option:
% \usepackage[preprint]{nips_2018}

% to compile a camera-ready version, add the [final] option, e.g.:
% \usepackage[final]{nips_2018}

% to avoid loading the natbib package, add option nonatbib:
% \usepackage[nonatbib]{nips_2018}

\usepackage[utf8]{inputenc} % allow utf-8 input
\usepackage[T1]{fontenc}    % use 8-bit T1 fonts
\usepackage{hyperref}       % hyperlinks
\usepackage{url}            % simple URL typesetting
\usepackage{booktabs}       % professional-quality tables
\usepackage{amsfonts}       % blackboard math symbols
\usepackage{nicefrac}       % compact symbols for 1/2, etc.
\usepackage{microtype}      % microtypography
\usepackage{bm}
\usepackage{amsmath}
\usepackage{amsthm}
\usepackage{amssymb,amsfonts}
\newcommand{\argmax}{\operatornamewithlimits{arg\,max}}
\newcommand{\argmin}{\operatornamewithlimits{arg\,min}}
\newtheorem{assumption}{Assumption}
\newtheorem{lemma}{Lemma}
\newtheorem{theorem}{Theorem}
\newtheorem{corollary}{Corollary}

%\usepackage[noend]{algpseudocode}
%\renewcommand{\algorithmicrequire}{\textbf{Input:}} % Use Input in the format of Algorithm
%\renewcommand{\algorithmicensure}{\textbf{Output:}} % Use Output in the format of Algorithm
\title{Max-Discrepancy Distributed Learning: Fast Risk Bounds and Algorithms}

% The \author macro works with any number of authors. There are two
% commands used to separate the names and addresses of multiple
% authors: \And and \AND.
%
% Using \And between authors leaves it to LaTeX to determine where to
% break the lines. Using \AND forces a line break at that point. So,
% if LaTeX puts 3 of 4 authors names on the first line, and the last
% on the second line, try using \AND instead of \And before the third
% author name.

\author{
  %David S.~Hippocampus\thanks{Use footnote for providing further
%    information about author (webpage, alternative
%    address)---\emph{not} for acknowledging funding agencies.} \\
%  Department of Computer Science\\
%  Cranberry-Lemon University\\
%  Pittsburgh, PA 15213 \\
%  \texttt{hippo@cs.cranberry-lemon.edu} \\
  %% examples of more authors
  %% \And
  %% Coauthor \\
  %% Affiliation \\
  %% Address \\
  %% \texttt{email} \\
  %% \AND
  %% Coauthor \\
  %% Affiliation \\
  %% Address \\
  %% \texttt{email} \\
  %% \And
  %% Coauthor \\
  %% Affiliation \\
  %% Address \\
  %% \texttt{email} \\
  %% \And
  %% Coauthor \\
  %% Affiliation \\
  %% Address \\
  %% \texttt{email} \\
}

\begin{document}
% \nipsfinalcopy is no longer used

\maketitle

\begin{abstract}
  We study the risk performance of distributed learning for the regularization empirical
  risk minimization with fast convergence rate,
  substantially improving the existing divide-and-conquer based distributed learning error analysis.
  An interesting theoretical finding is that the larger the discrepancy of each local estimate is, the tighter the risk bound is.
  This theoretical analysis motivates us to devise an effective max-discrepancy distributed learning algorithm (MDD).
  Experimental results show that our proposed method can outperform the existing divide-and-conquer methods
  but with little additional time cost.
  Theoretical analysis and empirical results demonstrate that our MDD  is sound and effective.
\end{abstract}
\section{Introduction}
In the era of big data, the rapid expansion of computing capacities in automatic data generation
and acquisition brings data of unprecedented size and complexity, and raises a series
of scientific challenges such as storage bottleneck and algorithmic scalability \cite{zhou2014big,Zhang2013,lin2017distributed}.
%There are some approaches for
%When the size of the dataset becomes extremely large, however, it may be infeasible
%to store all of the data on a single computer, or at least to keep the data in memory.
Distributed learning based on a divide-and-conquer approach has triggered enormous
recent research activities in various areas such as optimization \cite{zhang2012communication}
data mining \cite{wu2014data} and machine learning \cite{gillick2006mapreduce}.
This learning strategy breaks up a big problem into manageable
pieces, operates learning algorithms on each piece on individual machines or
processors, and then puts the individual solutions together to get a final global output.
In this way, distributed learning is a feasible technique to conquer big data challenges.
%Distributed learning a feasible method to overcome the difficulty.
%The average mixture algorithm perhaps the simplest algorithm for distributed statistical inference.
%The algorithm is appealing in its simplicity: partition the dataset $\mathcal{S}$ of size $N$ randomly into $m$ equal sized
%subsets $\mathcal{S}_i$, and we compute the estimate for each of the $i=1,\ldots,m$ subsets independently,
%and finally compute the average of partition-based estimate.
%Theoretical foundations of distributed learning form a hot topic in machine learning
%and have been explored recently in the framework of learning theory \cite{zhang2012communication,Zhang2013,lin2017distributed,guo2017learning}.

This paper aims at error analysis of the distributed learning for (regularization) empirical risk minimization.
Given $\mathcal{S}=\left\{z_i=(\mathbf  x_i,y_i)\right\}_{i=1}^N \in (\mathcal{Z}=\mathcal{X}\times \mathcal{Y})^N$,
drawn identically and independently from a fixed,
but unknown probability  distribution $\mathbb{P}$ on
$\mathcal{Z}=\mathcal{X}\times\mathcal{Y}$,
the (regularization) empirical risk minimization can be stated as
\begin{align}
\label{def-empirical-f}
  \hat{f}=\argmin_{f\in\mathcal{H}} \hat{R}(f):=\frac{1}{N}\sum_{j=1}^N\ell(f,z_j)+r(f)
\end{align}
where $\ell(f,z)$ is a loss function, $r(f)$ is a regularizer, and $\mathcal{H}$ is a hypothesis space.
This learning algorithm  has been well studied in learning theory,
see e.g. \cite{de2005model,caponnetto2007optimal,steinwart2009optimal,smale2007learning,steinwart2008support}.
The distributed learning algorithm studied in this paper starts with partitioning the
data set $\mathcal{S}$ into $m$ disjoint subsets $\{\mathcal{S}_i\}_{i=1}^m$, $|\mathcal{S}_i|=\frac{N}{m}=:n$.
Then it assigns each data subset $\mathcal{S}_i$ to one
machine or processor to produce a local estimator $\hat{f}_i$:
\begin{align*}
  \hat{f}_i=\argmin_{f\in\mathcal{H}}\hat{R}_i(f):=
    \frac{1}{|\mathcal{S}_i|}\sum_{z_j\in\mathcal{S}_i}\ell(f,z_j)+r(f).
\end{align*}
%Recall \cite{zhang2012communication}
The finally  global estimator $\bar{f}$ is synthesized by
$\bar{f}=\frac{1}{m}\sum_{i=1}^m\hat{f}_i.$

Theoretical foundations of distributed learning form a hot topic in machine learning
and have been explored recently in the framework of learning theory \cite{zhang2012communication,Zhang2013,lin2017distributed,guo2017learning}.
%This algorithm has been studied with a matrix analysis approach in \cite{zhang2012communication,Zhang2013}.
Under local strong convexity, smoothness and a reasonable set of other conditions, \cite{zhang2012communication} showed that the mean-squared error
decays as
\begin{align*}
  \mathbb{E}\left[\left\|\bar{f}-f^\ast\right\|^2\right]=\mathcal{O}\left(\frac{1}{N}+\frac{1}{n^2}\right),
\end{align*}
where $f^\ast$ is the optimal hypothesis in the hypothesis space.
Under some eigenfunction assumption,
the error analysis for distributed regularized
least squares in reproducing kernel Hilbert space (RKHS) was established in \cite{Zhang2013}:
if $m$ is not too large,
\begin{align*}
  \mathbb{E}\left[\left\|\bar{f}-f^\ast\right\|^2\right]=\mathcal{O}\left(\|f_\ast\|_\mathcal{H}^2+\frac{\gamma(\lambda)}{N}\right),
\end{align*}
where $\gamma(\lambda)=\sum_{j=1}^\infty\frac{\mu_j}{\lambda+\mu_j}$,
$\mu_j$ is the eigenvalue of a Mercer kernel function.
Without any eigenfunction assumption,
 an improved bound was derived for some $1\leq p\leq\infty$ \cite{lin2017distributed}:
\begin{align*}
  \mathbb{E}\left[\left\|\bar{f}-f^\ast\right\|_2\right]=
  \mathcal{O}\left(\left(\frac{\gamma(\lambda)}{N}\right)^{\frac{1}{2}(1-\frac{1}{p})}\left(\frac{1}{N}\right)^{\frac{1}{2p}}\right).
\end{align*}

There are two main contributions in this paper.
First, under strongly convex and smooth, and a reasonable set of other conditions,
we derive a risk bound:
\begin{align}
\label{theorail-fast-rate}
    R(\bar{f})-R(f_\ast)=\mathcal{O}\left(\frac{H_\ast}{n}
    +\frac{1}{n^2}
    -\Delta_{\bar{f}}\right),
  \end{align}
  where $R(f)=\mathbb{E}_{z}[\ell(f,z)]+r(f)$,
  $\Delta_{\bar{f}}=\mathcal{O}\left(\frac{1}{m^2}\sum_{i,j=1,i\not=j}^m\|\hat{f}_i-\hat{f}_j\|^2\right)$ is the discrepancy between all partition-based estimates and
  $H_\ast=\mathbb{E}_{z}\left[\ell(f_\ast,z)\right]$.
 % $\mathcal{N}(\mathcal{H},1/n)$ be the $1/n$-net of $\mathcal{H}$ with minimal cardinality,
%and $C(\mathcal{H},1/n)$ the covering number of $|\mathcal{N}(\mathcal{H},1/n)|$.
When the minimal risk is small, i.e., $H_\ast=\mathcal{O}\left(\frac{1}{n}\right)$,
the rate is improved to
\begin{align*}
    R(\bar{f})-R(f_\ast)=\mathcal{O}\left(\frac{1}{n^2}-\Delta_{\bar{f}}\right).
\end{align*}
Thus, if $m\leq \sqrt{N}$, the order of $R(\bar{f})-R(f_\ast)$ is faster than
$\mathcal{O}\left(\frac{1}{N}-\Delta_{\bar{f}}\right).$
Note that if $\ell(f,z)+r(f)$ is $L$-Lipschitz continuous over $f$,
the order of $R(\bar{f})-R(f^\ast)$ is
\begin{align*}
  R(\bar{f})-R(f^\ast)=\mathcal{O}\left(L\mathbb{E}\left[\left\|\bar{f}-f^\ast\right\|\right]\right)=\mathcal{O}\left(L\sqrt{\mathbb{E}\left[\left\|\bar{f}-f^\ast\right\|^2\right]}\right).
\end{align*}
Thus, the order of $R(\bar{f})-R(f^\ast)$ in \cite{zhang2012communication,Zhang2013,lin2017distributed}
 at most $\mathcal{O}\big({\frac{1}{\sqrt{N}}}\big)$,
 which is much slower than that of our bound $\mathcal{O}\left({\frac{1}{N}}\right)$.
Our second contribution is to develop a novel max-discrepancy distributed learning algorithm.
From Equation \eqref{theorail-fast-rate},
we know that the larger the discrepancy $\Delta_{\bar{f}}$ is, the tighter the risk bound is.
This  interesting theoretical finding motivates us to devise a max-discrepancy distributed learning algorithm (MDD):
\begin{align}
\label{equation-op}
  \hat{f}_i=\argmin_{f\in\mathcal{H}}\frac{1}{|\mathcal{S}_i|}\sum_{z_j\in\mathcal{S}_i}\ell(f,z_j)+r(f)-\gamma \|f-\bar{f}_{\backslash i}\|_\mathcal{H},
\end{align}
where $\bar{f}_{\backslash i}=\frac{1}{m-1}\sum_{j=1,j\not =i}^m\hat{f}_j$.
The last term of \eqref{equation-op} is to make $\Delta_{\bar{f}}$ large.
%We present a simple iterative algorithm to solve the above optimization problem.
Experimental results on lots of datasets show that our proposed MDD is sound and efficient.


The rest of the paper is organized as follows.
In Section 2, we derive a risk bound of distributed learning with fast convergence rate.
In Section 3, we  propose two novel algorithms based on the max-discrepancy of each local estimate in linear space and RKHS.
In Section 4, we empirically analyze the performance of our proposed algorithms.
We end in Section 5 with conclusion.
%Due to limited space,
%all the proofs are given in the supplementary material.


%\section{Preliminaries}
%We consider  the supervised learning where a learning algorithm receives a sample of $N$ labeled points
%\begin{align*}
%  \mathcal{S}=\left\{z_i=(\mathbf  x_i,y_i)\right\}_{i=1}^N
%        \in (\mathcal{Z}=\mathcal{X}\times \mathcal{Y})^N,
%\end{align*}
%where $\mathcal{X}$ denotes the input space and
%$\mathcal{Y}$ denotes the output space.
%%$\mathcal{Y}\subset \mathbb{R}$ in the regression case and
%%$\mathcal{Y}=\{-1,+1\}$ in classification case.
%We assume $\mathcal{S}$ is drawn identically and independently from a fixed,
%but unknown probability  distribution $\mathbb{P}$ on
%$\mathcal{Z}=\mathcal{X}\times\mathcal{Y}$.
%The goal is to learn a good prediction model $f\in\mathcal{H}:\mathcal{X}\rightarrow\mathcal{Y}$,
%whose prediction accuracy at instance
%$z=(\mathbf x,y)$ is measured by a loss function $\ell(f,z)$.
%
%
%In this paper, we focus on the (regularization) empirical risk minimization over some Hilbert space $\mathcal{H}$:
%\begin{align}
%  \hat{f}=\argmin_{f\in\mathcal{H}} \hat{R}(f)=\frac{1}{N}\sum_{j=1}^N\ell(f,z_j)+r(f)
%\end{align}
%where $\ell(f,z)$ is the loss function, and $r(f)$ is a regularizer.
%The expect (regularization) risk is defined as
%$$f^\ast=\argmin_{f\in\mathcal{H}}R(f)=\mathbb{E}_{z\sim \mathbb{P}}[\ell(f,z)]+ r(f).$$
%
%In the distributed setting, we divide evenly amongst $m$ processors or inference procedures.
%Let $\mathcal{S}_i, i\in(1,2,\ldots,m)$, denote a subsampled dataset of size $n=\frac{N}{m}$.
%For each $i=1,2,\ldots,m$, the local estimate
%\begin{align*}
%  \hat{f}_i=\argmin_{f\in\mathcal{H}}\hat{R}_i(f)=
%    \frac{1}{n}\sum_{z_j\in\mathcal{S}_i}\ell(f,z_j)+r(f).
%\end{align*}
%The average local estimates is denote as
%$
%  \bar{f}=\frac{1}{m}\sum_{i=1}^m\hat{f}_i.
%$

%In the next, we will estimate the discrepancy of $R(\bar{f})$ and $R(f^\ast)$.
\section{Error Analysis of Distributed Learning}
In this section, we will derive a sharper risk bound under some common assumptions.
\subsection{Assumptions}
In the following, we use $\|\cdot\|_\mathcal{H}$ to denote the norm induced by inner product of the Hilbert space $\mathcal{H}$.
Let the expected risk  $R(f)$ and $f_\ast$ be
\begin{align*}
  R(f)=\mathbb{E}_{z}[\ell(f,z)]+r(f) \text{ and } f_\ast=\argmin_{f\in\mathcal{H}}R(f).
\end{align*}
\begin{assumption}
\label{assumption-strongly}
  The risk $R(f)$ is an $\eta$-strongly convex function,
  that is $\forall f,f'\in\mathcal{H}$,
  \begin{align}
    \label{assumption-strongly-equation}
     \langle \nabla R(f), f-f'\rangle_\mathcal{H}+\frac{\eta}{2}\|f-f'\|_\mathcal{H} &\leq R(f)-R(f'),
  \end{align}
  or (another equivalent definition)
  $\forall  f,f'\in\mathcal{H}, t\in[0,1]$,
  \begin{align}
  \label{assumption-strongly-second}
  R(tf+(1-t)f') \leq  tR(f)+(1-t)R(f')-\frac{1}{2}\eta t(t-1)\|f-f'\|_\mathcal{H}^2.
  \end{align}
\end{assumption}
\begin{assumption}
\label{assumption-strongly-emprical}
  The empirical risk $\hat{R}(f)$  is a convex function.
  %that is $\forall f,f'\in\mathcal{H}$, $z\in\mathcal{Z}$,
%  \begin{align}
%    \label{assumption-strongly-equation}
%     \langle \nabla R(f), f-f'\rangle_\mathcal{H}+\frac{\eta'}{2}\|f-f'\|_\mathcal{H} &\leq R(f)-R(f')
%  \end{align}
\end{assumption}
\begin{assumption}
\label{assumption-smooth-loss}
  The loss function $\ell(f,z)$ is $\tau$-smooth with respect to the first variable $f$,
  that is $\forall f,f'\in\mathcal{H}$,
  \begin{align}
     \label{assumption-smooth-equaiton-loss}
     \left\|\nabla\ell(f,\cdot)-\nabla \ell(f',\cdot)\right\|_\mathcal{H}&\leq \tau\|f-f'\|_\mathcal{H}.
  \end{align}
\end{assumption}
\begin{assumption}
\label{assumption-smooth-r}
  The regularizer $r(f)$ is a $\tau'$-smooth function,
  that is $\forall f,f'\in\mathcal{H}$,
  \begin{align}
     \label{assumption-smooth-equaiton-loss}
     \left\|\nabla r(f)-\nabla r(f')\right\|_\mathcal{H}&\leq \tau'\|f-f'\|_\mathcal{H}.
  \end{align}
\end{assumption}
%\begin{assumption}
%\label{assumption-smooth}
%  The function $\nu(f,z)=\ell(f,z)+r(f)$ is $\tilde{\tau}$-smooth with respect to the first variable $f$,
%  that is $\forall f,f'\in\mathcal{H}$,
%  \begin{align}
%     \label{assumption-smooth-equaiton}
%     \left\|\nabla\nu(f,\cdot)-\nabla \nu(f',\cdot)\right\|_\mathcal{H}&\leq \tilde{\tau}\|f-f'\|_\mathcal{H}.
%  \end{align}
%\end{assumption}

\begin{assumption}
\label{assumption-libs}
  The function $\nu(f,z)=\ell(f,z)+r(f)$ is $L$-Lipschitz continuous with respect to the first variable $f$,
  that is $\forall f,f'\in\mathcal{H}$,
  \begin{align}
     \label{assumption-libs-equation}
     \left\|\nu(f,\cdot)- \nu(f',\cdot)\right\|_\mathcal{H}&\leq L\|f-f'\|_\mathcal{H}.
  \end{align}
\end{assumption}

\textbf{Assumptions} \ref{assumption-strongly}, \ref{assumption-strongly-emprical}, \ref{assumption-smooth-loss}, \ref{assumption-smooth-r} and \ref{assumption-libs} allow us to model some popular losses,
such as square loss and logistic loss, and some regularizer, such as $r(f)=\lambda \|f\|_\mathcal{H}^2$.

\begin{assumption}
\label{assumption-optimal-bound}
  We assume that the gradient at $f_\ast$ is upper bounded by $M$, that is
  \begin{align*}
    \|\nabla \ell(f^\ast,\cdot)\|_\mathcal{H}\leq M.
  \end{align*}
\end{assumption}
Assumption \ref{assumption-optimal-bound} is also a common assumption, which is used in \cite{Zhang2017er,zhang2012communication}.
\subsection{Faster Rate of Distributed Learning}
Let $\mathcal{N}(\mathcal{H},\epsilon)$ be the $\epsilon$-net of $\mathcal{H}$ with minimal cardinality,
and $C(\mathcal{H},\epsilon)$ the covering number of $|\mathcal{N}(\mathcal{H},\epsilon)|$

\begin{theorem}
\label{theorem-main}
For any $0<\delta<1$, $\epsilon\geq 0$,
under \textbf{Assumptions}  \ref{assumption-strongly}, \ref{assumption-strongly-emprical},
\ref{assumption-smooth-loss}, \ref{assumption-smooth-r}, \ref{assumption-libs} and \ref{assumption-optimal-bound},
and when
  \begin{align}
    \label{equation-12}
    m\leq \frac{N\eta}{4\tilde{\tau}\log C(\mathcal{H},\epsilon)},
  \end{align}
  with probability at least $1-\delta$,
  we have
  \begin{align}
    \label{equation-13}
    \begin{aligned}
    R(\bar{f})-R(f_\ast)&\leq
    \frac{16\tilde{\tau} \log(4m/\delta)}{n^2\eta}+\frac{128\tau H_\ast\log(4m/\delta)}{n\eta}+\frac{32\tilde{\tau}^2\epsilon^2}{\eta}+
    \frac{64\tilde{\tau} L \epsilon \log C(\mathcal{H},\epsilon)}{n\eta}\\
   &~~~~~~+\frac{64\tilde{\tau} \epsilon^2 \log^2C(\mathcal{H},\epsilon)}{n^2\eta}
   -\Delta_{\bar{f}},
  \end{aligned}
  \end{align}
  where $\Delta_{\bar{f}}=\frac{\eta}{4m^2}\sum_{i,j=1,i\not=j}^m\|\hat{f}_i-\hat{f}_j\|_\mathcal{H}^2$, $H_\ast=\mathbb{E}_{z}\left[\ell(f_\ast,z)\right]$ and $\tilde{\tau}=\tau+\tau'$.
\end{theorem}
From the above theorem, an  interesting finding is that,
when the larger the discrepancy of each local estimate is,
the tighter the risk bound is.
Furthermore, one can also see that when $\epsilon$ small enough,
$$\frac{32\tilde{\tau}^2\epsilon^2}{\eta}+
    \frac{64\tilde{\tau} L \epsilon \log C(\mathcal{H},\epsilon)}{n\eta}
    +\frac{64\tilde{\tau} \epsilon^2\log^2C(\mathcal{H},\epsilon)}{n^2\eta}$$
will becomes non-dominating.
To be specific, we have the following corollary:
\begin{corollary}
\label{corollary-first}
  By setting $\epsilon=\frac{1}{n}$ in Theorem \ref{theorem-main},
  when $m\leq \frac{N\eta}{4\tilde{\tau}\log C(\mathcal{H},1/n)}$,
  with high probability,
  we have
  \begin{align*}
    R(\bar{f})-R(f_\ast)=\mathcal{O}\left(\frac{H_\ast\log(m)}{n}
    +\frac{\log(\mathcal{N}(\mathcal{H},\frac{1}{n}))}{n^2}
    -\Delta_{\bar{f}}\right).
  \end{align*}
\end{corollary}
If the the minimal risk $H_\ast$ is small, i.e., $H_\ast=\mathcal{O}(\frac{1}{n})$,
the rate can even reach $$\mathcal{O}\left(\frac{\log(m)}{n^2}
    +\frac{\log(\mathcal{N}(\mathcal{H},\frac{1}{n}))}{n^2}
    -\Delta_{\bar{f}}\right).$$
To the best of our knowledge,
this is the first $\tilde{\mathcal{O}}\left(\frac{1}{n^2}\right)$-type of distributed
risk bound for (regularization) empirical risk minimization.

In the next, we will consider two popular Hilbert spaces:
linear and reproducing kernel Hilbert space.
\subsubsection{Linear Space}
\label{subsection-3.1}
The linear hypothesis space we considered is defined as
\begin{align*}
\mathcal{H}=\left\{f=\mathbf w^\mathrm T\mathbf x\Big|\mathbf w\in \mathbb{R}^d, \|\mathbf w\|_2\leq B\right\}.
\end{align*}
From \cite{pisier1999volume},
the cover number of linear hypothesis space can be bounded by
\begin{align*}
  \log\left(C(\mathcal{H},\epsilon)\right)\leq d\log \left(\frac{6B}{\epsilon}\right).
\end{align*}
Thus, if we set $\epsilon=\frac{1}{n}$, from Corollary \ref{corollary-first}, we have
\begin{align*}
  R(\bar{f})-R(f_\ast)&=\mathcal{O}\left(\frac{H_\ast\log m}{n}+\frac{d\log n}{n^2}-
  \Delta_{\bar{f}}\right)
  %&\leq \mathcal{O}\left(\frac{1}{n^2}+\frac{R_\ast\log m}{n}+\frac{d\log n}{n^2}\right)
\end{align*}
When the minimal risk is small, i.e., $H_\ast=\mathcal{O}\left(\frac{d}{n}\right)$,
the rate is improved to
\begin{align*}
    \mathcal{O}\left(\frac{d\log (mn)}{n^2}-\Delta_{\bar{f}}\right)=\mathcal{O}\left(\frac{d\log N}{n^2}-\Delta_{\bar{f}}\right).
\end{align*}
Therefore, if $m\leq \sqrt{\frac{N}{d\log N}}$, the order of risk bound can even faster than
$\mathcal{O}\left(\frac{1}{N}\right).$
\subsubsection{Reproducing Kernel Hilbert Space}
\label{subsection-3.2}
The reproducing kernel Hilbert space $\mathcal{H}_K$ associated with the kernel $K$ is
defined to be the closure of the linear span of the set of functions
$\left\{K(\mathbf x,\cdot):\mathbf x\in\mathcal{X}\right\}$ with the inner product satisfying
\begin{align*}
  \langle K(\mathbf x,\cdot), f\rangle_{K}=f(\mathbf x), \forall \mathbf x\in\mathcal{X}, f\in\mathcal{H}_K.
\end{align*}

The  hypothesis space of the reproducing kernel Hilbert space we considered in this paper is
\begin{align*}
  \mathcal{H}:=\left\{f\in\mathcal{H}_{K}: \|f\|_{K}\leq B\right\}.
\end{align*}

From \cite{zhou2002covering}, if the kernel function $K$ is the popular Gaussian kernel over $[0,1]^d$:
$$
  K(\mathbf x,\mathbf x')=\exp\left\{-\frac{\|\mathbf x-\mathbf x'\|^2}{\sigma^2}\right\}, \mathbf x,\mathbf x' \in[0,1]^d,
$$
then for $0\leq \epsilon\leq \frac{B}{2}$,
$$
 \log \left(C(\mathcal{H},\epsilon)\right)=\mathcal{O}\left(\log^d\left(\frac{B}{\epsilon}\right)\right).$$
From Corollary \ref{corollary-first}, if we set $\epsilon=\frac{1}{n}$, and assume $R_\ast=\mathcal{O}\left(\frac{1}{n}\right)$,
we have
\begin{align*}
  R(\bar{f})-R(f_\ast)=\mathcal{O}\left(\frac{\log m}{n^2}+\frac{\log^d n}{n^2}-
  \Delta_{\bar{f}}\right)
\end{align*}
Therefore, if $m\leq \min\left\{\sqrt{\frac{N}{d\log N}}, \sqrt{\frac{N}{\log^d n}}\right\}$,
the order is faster than $\mathcal{O}\left(\frac{1}{N}\right)$.
%Note that for our bound,
%\begin{align*}
%     R(f)-R(f_\ast)=\left\{
%     \begin{aligned}
%     &\mathcal{O}\left(\frac{d\log N}{n^2}-\Delta_{\bar{f}}\right)  &&\text{linear hypothesis space}\\
%     &\mathcal{O}\left(\frac{\log^d n}{n^2}-\Delta_{\bar{f}}\right) &&\text{RKHS  space}
%     \end{aligned}
%     \right.
%\end{align*}
%So, we can obtain that
%\begin{align*}
%     R(f)-R(f_\ast)=\mathcal{O}\left(\frac{1}{N}-\Delta_{\bar{f}}\right)
%     % \left\{
%%     \begin{aligned}
%%     &\mathcal{O}\left(\frac{1}{N}-\Delta_{\bar{f}}\right)  &\text{if} m\leq \sqrt{\frac{N}{d\log N}} &\text{for linear hypothesis space}\\
%%     &\mathcal{O}\left(\frac{\log^d n}{n^2}-\Delta_{\bar{f}}\right) &ff &\text{RKHS  space}
%%     \end{aligned}
%%     \right.
%\end{align*}
%if $m\leq \sqrt{\frac{N}{d\log N}}$ for linear hypothesis space, or $m\leq \sqrt{\frac{N}{\log^2 n}}$ for RKHS space.

\subsection{Comparison with Related Work}
In this subsection, we will compare our bound with the related work \cite{zhang2012communication,Zhang2013,lin2017distributed}.
Under the smooth, strongly convex and other some assumptions,
a distributed risk bound is given in \cite{zhang2012communication}:
\begin{align*}
  \mathbb{E}\left[\|\bar{f}-f_\ast\|^2\right]=\mathcal{O}\left(\frac{1}{N}+\frac{\log d}{n^2}\right).
\end{align*}
Under some eigenfunction assumption,
the error analysis for distributed regularized
least squares were established in \cite{Zhang2013},
\begin{align*}
  \mathbb{E}\left[\left\|\bar{f}-f^\ast\right\|^2\right]=\mathcal{O}\left(\|f_\ast\|_\mathcal{H}^2+\frac{\gamma(\lambda)}{N}\right).
\end{align*}
%where $\gamma(\lambda)=\sum_{j=1}^\infty\frac{\mu_j}{\lambda+\mu_j}$,
%$\mu_j$ is the eigenvalue of a Mercer kernel function.
%where $r$ is the rank of the kernel function.
By removing the eigenfunction assumptions with a novel integral operator method of \cite{Zhang2013},
 a new bound was derived \cite{lin2017distributed}:
\begin{align*}
  \mathbb{E}\left[\left\|\bar{f}-f^\ast\right\|\right]=
  \mathcal{O}\left(\left(\frac{\gamma(\lambda)}{N}\right)^{\frac{1}{2}(1-\frac{1}{p})}\left(\frac{1}{N}\right)^{\frac{1}{2p}}\right).
\end{align*}
%of $\mathbb{E}\left[\left\|\bar{f}-f^\ast\right\|^2\right]$ of
%order at most $\mathcal{O}\left(\frac{1}{N}\right)$.
%and if $m$ is not very large,
%they show that
%\begin{align*}
%   \mathbb{E}\left[\|\bar{f}-f_\ast\|^2\right]=\mathcal{O}\left(\frac{1}{N}\right).
%\end{align*}
If $\nu(f,z)$ is $L$-Lipschitz continuous over $f$, that is
\begin{align*}
  \forall f, f\in \mathcal{H}, z\in\mathcal{Z}, |\nu(f,z)-\nu(f',z)|\leq L\|f-f'\|,
\end{align*}
it is easy to verity that
\begin{align}
  \nonumber R(f)-R(f_\ast)&\leq L\mathbb{E}\left[\|\bar{f}-f_\ast\|\right]\leq L\sqrt{\mathbb{E}\left[\|\bar{f}-f_\ast\|^2\right]}
  %\label{related-work-one}    &= \mathcal{O}\left(\frac{1}{\sqrt{N}}+\frac{\sqrt{\log d}}{n}\right).
\end{align}
Thus, the order of \cite{zhang2012communication,Zhang2013,lin2017distributed} of  $R(f)-R(f_\ast)$ is at most $\mathcal{O}\left(\frac{1}{\sqrt{N}}\right)$.

According to the subsections \ref{subsection-3.1} and \ref{subsection-3.2},
if $m$ is not very large, and $H_\ast$ is small,
the order of this paper can even faster than $\mathcal{O}\left(\frac{1}{N}\right)$,
which is much faster than those of the related work \cite{zhang2012communication,Zhang2013,lin2017distributed}.

\section{Max-Discrepant Distributed Learning (MDD)}
In this section, we will propose two novel algorithms for linear space and RKHS.
From corollary \ref{corollary-first},  we know that
  $
     R(f)-R(f_\ast)=\mathcal{O}\Big(\frac{1}{n^2}-\frac{1}{m^2}\sum_{i,j=1,i\not=j}^m\|\hat{f}_i-\hat{f}_j\|_\mathcal{H}^2\Big).
$
Thus, to obtain tighter bound, the discrepancy of each local estimate $\hat{f}_i, i=1,\ldots,m$ should be larger.
%In the next, we will propose two algorithms for linear space and RKHS.

\begin{algorithm}
    \caption{Max-Discrepant Distributed Learning for Linear Space (MDD-LS)}
    \label{alg:RMMls}
    \begin{algorithmic}[1]
    \STATE \textbf{Input}: $\lambda,\gamma$, $\mathbf X$, $m$, $\zeta$.
     \STATE \emph{For each worker node $i$:} $\hat{\mathbf w}_i^0=\mathbf A_i^{-1} \mathbf b_i$, and push $\hat{\mathbf w}_i^0$ to the server node.\\
         ~~~~~~~~// $\mathbf A_i=\frac{1}{n}\mathbf X_{\mathcal{S}_i}\mathbf X_{\mathcal{S}_i}^\mathrm{T}+
            \lambda \mathbf I_d$, $\mathbf b_i= \frac{1}{n}\mathbf X_{\mathcal{S}_i}\mathbf y_{\mathcal{S}_i}$.
           % ~~~~~~~~// for RKHS $\mathbf A_i=\frac{1}{n}\mathbf K_{\mathcal{S}_i}+
%            \lambda \mathbf I_n-\gamma \mathbf I_n$, $\mathbf b_i= \frac{1}{n}\mathbf y_{\mathcal{S}_i}$
    \STATE \emph{For server node:}
    $\bar{\mathbf w}^0=\frac{1}{m}\sum_{i=1}^m\hat{\mathbf w}_i^0$,
    $\bar{\mathbf w}^{0}_{\backslash i}=\frac{m\bar{\mathbf w}^{0}-\hat{\mathbf w}_i^0}{m-1}$.
    \FOR{$t=1,2,\ldots$}
    \STATE  \emph{For each worker node $i$:} \\
    ~~~~Pull $\bar{\mathbf w}^{t-1}_{\backslash i}$ from server node.
    \STATE ~~~~$\mathbf d_i^t=\left(\left(\bar{\mathbf w}^{t-1}_{\backslash i}\right)^\mathrm{T}\hat{\mathbf w}_i^{0}\right)./\mathbf b_i$.
     ~~~~$\hat{\mathbf w}_i^t=\hat{\mathbf w}_i^0-\gamma\mathbf d_i^t$.
    \STATE ~~~~Push $\hat{\mathbf w}_i^t$ to the server node.
     \STATE \emph{For server node:}
     \STATE ~~~~$\bar{\mathbf w}^t=\frac{1}{m}\sum_{i=1}^m\hat{\mathbf w}_i^t$\\
      ~~~~\textbf{if} {$\|\bar{\mathbf w}^{t}-\bar{\mathbf w}^{t-1}\|\leq \zeta$} \textbf{end for}
      \STATE ~~~~\textbf{else} $\bar{\mathbf w}^{t}_{\backslash i}=\frac{m\bar{\mathbf w}^{t}-\hat{\mathbf w}_i^t}{m-1}$.
    \ENDFOR
    \STATE \textbf{Output}: $\bar{\mathbf w}=\frac{1}{m}\sum_{i=1}^m\hat{\mathbf w}_i^t$.
    \end{algorithmic}
\end{algorithm}
\subsection{Linear Hypothesis Space}
When $\mathcal{H}$ is a linear Hypothesis space,
we consider the following optimization problem:
\begin{align}
 \label{optimation-linear-space}
  \hat{\mathbf w}_i=\argmin_{\mathbf w\in\mathbb{R}^d}
  \frac{1}{n}\sum_{z_i\in\mathcal{S}_i}(\mathbf w^\mathrm{T}\mathbf x_i-y_i)^2+\lambda \|\mathbf w\|_2^2+ \gamma \mathbf w^\mathrm{T}\bar{\mathbf w}_{\backslash i},
\end{align}
where $\bar{\mathbf w}_{\backslash i}=\frac{1}{m-1}\sum_{j=1,j\not =i}\hat{\mathbf w}_j$.
Note that, if given $\bar{\mathbf w}_{\backslash i}$,  $\hat{\mathbf w}_i$ has following closed form solution:
\begin{align*}
  \hat{\mathbf w}_i=\Big(\underbrace{\frac{1}{n}\mathbf X_{\mathcal{S}_i}\mathbf X_{\mathcal{S}_i}^\mathrm{T}+\lambda \mathbf I_d}_{:=\mathbf A_i}\Big)^{-1}
  \Big(\underbrace{\frac{1}{n}\mathbf X_{\mathcal{S}_i}\mathbf y_{\mathcal{S}_i}}_{:=\mathbf b_i}- \frac{\gamma\bar{\mathbf w}_{\backslash i}}{2}\Big),
\end{align*}
where $\mathbf X_{\mathcal{S}_i}=(\mathbf x_{t_1},\mathbf x_{t_2},\ldots, \mathbf x_{t_n})$,
$\mathbf y_{\mathcal{S}_i}=(y_{t_1},y_{t_2},\ldots,y_{t_n})^\mathrm{T}$, $z_{t_j}\in \mathcal{S}_i$, $j=1,\ldots, n$.
In the next, we will give a iterative algorithm to
solve the optimization problem \eqref{optimation-linear-space}.
In each iteration, we should compute $\mathbf A_i^{-1}\bar{\mathbf w}_{\backslash i}$,
which needs $\mathcal{O}\left(d^2\right)$ if given $\mathbf A_i^{-1}$,
which is computational intensive.
%where $\mathbf A_i=\frac{1}{n}\mathbf X_{\mathcal{S}_i}\mathbf X_{\mathcal{S}_i}^\mathrm{T}+
%\lambda \mathbf I_d-\gamma \mathbf I_d$.
Fortunately, %the Lemma 4 (see in Appendix) show that we only need $\mathcal{O}(d)$ to compute $\mathbf A_i^{-1}\bar{\mathbf w}_{\backslash i}$
%if Given $\mathbf A_i^{-1}$.
%Let $\mathbf b_i=\frac{1}{n}\mathbf X_{\mathcal{S}_i}\mathbf y_{\mathcal{S}_i}$, $\mathbf c_i=\mathbf A_i^{-1}\mathbf b_i$,
from Lemma 4  (see in Appendix), the $\mathbf A_i^{-1}\bar{\mathbf w}_{\backslash i}$ can be computed by
\begin{align*}
  \mathbf A_i^{-1}\bar{\mathbf w}_{\backslash i}=
  \left(\bar{\mathbf w}_{\backslash i}^\mathrm{T}\mathbf c_i\right)./\mathbf b_i, \mathbf c_i=\mathbf A_i^{-1}\mathbf b_i
\end{align*}
where $a./\mathbf c=(a/c_1,\ldots a/c_d)^\mathrm{T}$, which only needs $\mathcal{O}(d)$.

The Max-Discrepant Distributed Learning  algorithm for linear space is given in Algorithm \ref{alg:RMMls}.
Compared with the traditional divide-and-conquer method,
our MDD for linear space only need add $\mathcal{O}(d)$ in each iteration for each worker node.
%We can see that in each iteration, the time complexity is $O()$
\begin{algorithm}
    \caption{Max-Discrepant Distributed Learning for RKHS (MDD-RKHS)}
    \label{alg:RMMRKHS}
    \begin{algorithmic}[1]
    \STATE \textbf{Input}: $\lambda,\gamma$, kernel function $K$, $\mathbf X$, $m$, $\zeta$.
     \STATE \emph{For each worker node $i$:} $\hat{\mathbf w}_i^0=\mathbf A_i^{-1} \mathbf b_i$, and push $\hat{\mathbf w}_i^t$ to the server node.\\
        % ~~~~~~~~// for linear space $\mathbf A_i=\frac{1}{n}\mathbf X_{\mathcal{S}_i}\mathbf X_{\mathcal{S}_i}^\mathrm{T}+
%            \lambda \mathbf I_d-\gamma \mathbf I_d$, $\mathbf b_i= \frac{1}{n}\mathbf X_{\mathcal{S}_i}\mathbf y_{\mathcal{S}_i}$\\
            ~~~~~~~~// $\mathbf A_i=\frac{1}{n}\mathbf K_{\mathcal{S}_i}+
            \lambda \mathbf I_n$, $\mathbf b_i= \frac{1}{n}\mathbf y_{\mathcal{S}_i}$.
    \STATE \emph{For server node:} $\hat{\mathbf g}_{i,j}^0=\mathbf K_{\mathcal{S}_i,\mathcal{S}_j}\hat{\mathbf w}_j^0$, $i, j=1,\ldots,m$, $\bar{\mathbf g}^{0}_{\backslash i}=\frac{m\bar{\mathbf g}_i^{0}-\hat{\mathbf g}_i^0}{m-1}$.
    \FOR{$t=1,2,\ldots$}
    \STATE  \emph{For each worker node $i$:}
    \STATE ~~~~Pull $\bar{\mathbf g}^{t-1}_{\backslash i}$ from server node.
    \STATE ~~~~$\mathbf d_i^t=\left(\left(\bar{\mathbf g}^{t-1}_{\backslash i}\right)^\mathrm{T}\hat{\mathbf w}_i^0\right)./\mathbf b_i$,
     $\hat{\mathbf w}_i^t=\hat{\mathbf w}_i^0-\gamma\mathbf d_i^t$.
    \STATE ~~~~Push $\hat{\mathbf w}_i^t$ to the server node.
     \STATE \emph{For server node:}
     \STATE ~~~~$\hat{\mathbf g}_{i,j}^{t}=\mathbf K_{\mathcal{S}_i,\mathcal{S}_j}\hat{\mathbf w}_j^t$, $i, j=1,\ldots,m$,
     $\bar{\mathbf g}_{i}^t=\frac{1}{m}\sum_{j=1}^m \hat{\mathbf g}_{i,j}^t$.\\
      ~~~~\textbf{if} {$\frac{1}{m}\sum_{i=1}^m\|\bar{\mathbf g}_i^{t}-\bar{\mathbf g}_i^{t}\|\leq \zeta$} \textbf{end for}
      \STATE ~~~~\textbf{else}
      \STATE ~~~~~~~~$\bar{\mathbf g}^{t}_{\backslash i}=\frac{m\bar{\mathbf g}_i^{t}-\hat{\mathbf g}_i^t}{m-1}$.
    \ENDFOR
    \STATE \textbf{Output}: $\bar{f}=\frac{1}{m}\sum_{i=1}^m\hat{f}_i$, where $\hat{f}_i=\mathbf k_{\mathcal{S}_i}^\mathrm{T}\hat{\mathbf w}_i$,
    where $\mathbf k_{\mathcal{S}_i}=(K(\mathbf x_1,\cdot),\ldots,K(\mathbf x_n,\cdot))^\mathrm{T}$, $z_j\in\mathcal{S}_i$
    \end{algorithmic}
\end{algorithm}

\subsection{Reproducing Kernel Hilbert Space}
When $\mathcal{H}$ is a reproducing kernel Hilbert space, that is $f(\mathbf x)=\sum_{j=1}^n w_j K(\mathbf x_j,\mathbf x)$,
we consider the following optimization problem:
\begin{align}
\label{opti-RKHS}
  \hat{\mathbf w}_i=\argmin_{\mathbf w\in\mathbb{R}^n}
  \frac{1}{n}\|\mathbf K_{\mathcal{S}_i}\mathbf w-\mathbf y_{\mathcal{S}_i}\|_2^2+\lambda \mathbf w^\mathrm{T}\mathbf K_{\mathcal{S}_i}\mathbf w
  +\frac{\gamma}{m-1} \sum_{j=1,j\not=i}^{m}\mathbf w^\mathrm{T}\mathbf K_{\mathcal{S}_i}\mathbf K_{\mathcal{S}_i,\mathcal{S}_j}\hat{\mathbf w}_j,
  %\mathbf K_{\mathcal{S}_i}\left(\mathbf w- \bar{\mathbf w}_{\backslash i}\right),
\end{align}
%where $\bar{\mathbf w}_{\backslash i}=\frac{1}{m-1}\sum_{j=1,j\not =i}\hat{\mathbf w}_j$ and
where $\mathbf K_{\mathcal{S}_i}=\Big[K(\mathbf x_{t_j},\mathbf x_{t_{j'}})\Big]_{j,j'=1}^n$,
$z_{t_j},z_{t_{j'}}\in\mathcal{S}_i$,
$\mathbf K_{\mathcal{S}_i,\mathcal{S}_j}=\Big[K(\mathbf x_{t_j},\mathbf x_{t_{k}})\Big]_{j,k=1}^n$,
$z_{t_j}\in\mathcal{S}_i,z_{t_{k}}\in\mathcal{S}_j$.
It is easy to verity that
$\hat{\mathbf w}_i$ can be written as
\begin{align*}
  \hat{\mathbf w}_i=\Big(\underbrace{\frac{1}{n}\mathbf K_{\mathcal{S}_i}+\lambda \mathbf I_n}_{:=\mathbf A_i}\Big)^{-1}
  \Big(\underbrace{\frac{1}{n}\mathbf y_{\mathcal{S}_i}}_{:=\mathbf b_i}- \frac{\gamma}{2}\bar{\mathbf g}_{\backslash i}\Big).
\end{align*}
%Let $\mathbf A_i=\mathbf K_{\mathcal{S}_i}+\lambda \mathbf I_n-\gamma\mathbf I_n$, $\mathbf b_i=\mathbf y_{\mathcal{S}_i}$.
where $\mathbf g_j=\mathbf K_{\mathcal{S}_i,\mathcal{S}_j}\hat{\mathbf w}_j$ and $\bar{\mathbf g}_{\backslash i}=\frac{1}{m-1}\sum_{j=1,j\not=i}^m \hat{\mathbf g}_j$.

Similar with the linear space, we need to compute $\mathbf A_i^{-1}\bar{\mathbf g}_{\backslash i}$ in each iterative.
From Lemma 4 (see in Appendix), we know that
\begin{align*}
  \mathbf A_i^{-1}\bar{\mathbf g}_{\backslash i}=
  \left(\bar{\mathbf g}_{\backslash i}^\mathrm{T}\mathbf c_i\right)./\mathbf b_i,  \mathbf c_i=\mathbf A_i^{-1}\mathbf b_i.
\end{align*}
The Max-Discrepant Distributed Learning algorithm for RKHS is also given in Algorithm \ref{alg:RMMRKHS}.
Compared with the traditional divide-and-conquer method,
our MDD for RKHS  only need add $\mathcal{O}(n)$ in each iteration for local machine.

\subsection{Complexity}
\textbf{Linear space}: At the very beginning, we need $\mathcal{O}\left(nd^2\right)$
to compute the $\mathbf A_i$,  $\mathcal{O}(d^3)$ to compute $\mathbf A_i^{-1}$ for each worker node.
In each iteration, worker nodes cost $\mathcal{O}(d)$ to compute $\mathbf d^t_i$ and
the server node costs $O(md)$ to compute $\bar{\mathbf w}^{t}_{\backslash i}$.
So, the sequential computation complexity is $\mathcal{O}\left(nd^2+d^3+Tmd\right)$, where $T$ is the number of iteration.
Moreover, the total communication complexity is $O(Td)$.

\textbf{RKHS}: At the very beginning, we need $\mathcal{O}\left(n^2d\right)$ to compute the $\mathbf A_i$
and $\mathcal{O}(n^3)$ to compute $\mathbf A_i^{-1}$.
In each iteration, worker nodes cost $\mathcal{O}(n)$ to compute $\mathbf d^t_i$
and the server node costs $O(mn)$ to compute $\bar{\mathbf g}^{t}_{\backslash i}$.
So, the sequential computation complexity is $\mathcal{O}\left(n^2d+n^3+Tmn\right)$, where $T$ is the number of iteration.
Moreover, the total communication complexity is $O(Tn)$.

\textbf{Divide-and-conquer approach}: The sequential complexities of linear space and RKHS
are $\mathcal{O}\left(nd^2+d^3\right)$ and $\mathcal{O}\left(n^2d+n^3\right)$, respectively.
Meanwhile, the communication complexities are $O(d)$ and $O(n)$.

%For the original non-distributed algorithm,
%the complexities of linear space and RKHS
%are $\mathcal{O}\left(\{Nd^2,mn^2d\}+md^3\right)$ and $\mathcal{O}\left(mn^2d+mn^3\right)$, respectively.

\section{Experiments}
In this section, we will compare our  \texttt{MDD} methods with the global method and divide-and-conquer method in both Linear Hypothesis and RKHS.
Actually, we compare six approaches: global Ridge Regression (\texttt{RR}) \cite{hoerl1970ridge},
divide-and-conquer Ridge Regression (\texttt{DRR}) and our \texttt{MDD-LS} (Algorithm \ref{alg:RMMls}) in Linear Hypothesis Space,
meanwhile, global Kernel Ridge Regression (\texttt{KRR}) \cite{an2007fast},
divide-and-conquer Kernel Ridge Regression (\texttt{KDRR}) \cite{Zhang2013} and our \texttt{MDD-RKHS} (Algorithm \ref{alg:RMMRKHS}) in Reproducing Kernel Hilbert Space.
Based on the recent distributed machine learning platform PARAMETER SERVER \cite{li2014scaling},
we implemented divide-and-conquer methods and \texttt{MDD} methods and do experiments on this framework.
%To evaluated performance of methods, we use 10 public available dataset from LIBSVM
%Data \footnote{Available at https://www.csie.ntu.edu.tw/~cjlin/libsvmtools/datasets/}.
%%The root Mean Square Error $E=\sqrt{\frac{1}{N}\sum_{i=1}^N{(f(x_i)-y_i)^2}}$ and run time are used as comparison principles.

We experiment on 10 publicly available datasets from LIBSVM data
%To evaluated performance of methods, we use 10 public available dataset from LIBSVM
\footnote{Available at https://www.csie.ntu.edu.tw/~cjlin/libsvmtools/datasets/}.
We run all methods on a computer node with 32 cores (2.40GHz) and 64 GB memory.
While global methods only use a single CPU core, distributed methods use all cores to simulate parallel environment.
For RKHS methods, we use the popular Gaussian kernels $K(\mathbf{x}, \mathbf{x}')=\exp(-{\|\mathbf{x}-\mathbf{x}'\|_2^2}/{2\sigma^2})$
as candidate kernels, and choose the best kernel from $\sigma \in \{2^i, i=-10, -9, \dots, 10\}$ by 5-folds cross-validation.
The regularized parameterized $\lambda \in \{10^i, i=-6,-5,\dots,3\}$ in all methods and $\gamma \in \{10^i, i=-6,-5,\dots,3\}$
in \texttt{MDD} methods are determined by 5-folds cross-validation on training data.
With the same kernel and parameters, for each data set,  we run all methods 30 times with random partitions on all data
sets of non-overlapping 70\% training data and 30\% testing data.

The root mean square error of all methods is reported in Table \ref{tabel:mse}. Meanwhile, we repeat divide-and-conquer methods on different amount of worker nodes, 5 and 10 for simplification. Furthermore, the statistical significance of difference between methods hold the best result and other methods are estimated by 30 times multiple training/testing splits. The result which has no statistical significant difference compared to the best one, is remarked by underline while the best results by bold.
The table can be summarized as follows:
1) Global methods outperform the distributed methods on all data sets;
2) Kernel methods can usually get more optimal results than that of Linear methods;
3) Some data sets are sensitive to data partition, whose results existing huge gap between global methods and distributed methods,
such as space\_ga, cpusmall and phishing, while others are not;
4) The increase of worker nodes causes higher root mean square error.
5) Our \texttt{MDD-LS} and \texttt{MDD-RKHS} exhibits better prediction accuracy than the $\texttt{DRR}$ and $\texttt{KDRR}$ in all cases, and can be comparable with global methods in some cases. This demonstrates the advantage of \texttt{MDD} methods in generalization performance.

\begin{table*}[t]
\small
\footnotesize
%\scriptsize
%\tiny
%\renewcommand{\captionfont}{\small}
   \caption{
    \small Comparison of average root mean square error of our \texttt{MDD-LS} and \texttt{MDD-RKHS} with
    \texttt{RR}, \texttt{DRR}, \texttt{KRR}, \texttt{DKRR}.
   %We set the numbers of our method (EPSVM) to be bold if our method outperforms all other methods (KTA, CKTA, FSM, 3-CV, 5-CV and 10-CV).
   }
   \label{tabel:mse}
    %\centering
    \begin{tabular*}{\linewidth}{@{\extracolsep{-0.10cm}}lccccccccc}
    \toprule
                                &madelon                  &space\_ga               &cpusmall            &phishing           &cadata             &a8a                  &a9a                    &cod\-rna                   &YearPred                 \\   \hline
RR                              &\textbf{0.971}           &\textbf{2.585}          &\textbf{45.150}     &\textbf{0.247}     &\textbf{1.932}     &\textbf{0.671}       &\textbf{0.673}         &\textbf{0.841}             &\textbf{12.233} \\
DRR-5                           &0.989                    &2.814                   &53.114              &0.262              &2.659              &\underline{0.681}    &\underline{0.680}      &0.855                      &14.216 \\
DRR-10                          &1.408                    &2.983                   &55.557              &0.273              &2.839              &0.725                &0.696                  &0.863                      &15.780 \\
MDD-LS-5                        &\underline{0.977}        &2.677                   &46.184              &0.257              &\underline{2.114}  &\underline{0.677}    &\underline{0.673}      &0.847                      &\underline{12.303}\\
MDD-LS-10                       &1.121                    &2.750                   &48.956              &0.268              &2.352              &0.703                &0.685                  &0.854                      &14.158\\
\hline \hline
KRR                             &\textbf{0.959}           &\textbf{1.458}          &\textbf{53.993}     &\textbf{0.167}     &\textbf{1.504}     &\textbf{0.659}       &\textbf{0.790}         &\textbf{0.671}             &/ \\
KDRR-5                          &1.142                    &2.389                   &\underline{54.228}  &0.419              &1.598              &0.873                &0.866                  &0.674                      &\underline{5.397}\\
KDRR-10                         &1.374                    &2.531                   &56.233              &0.422              &1.824              &0.906                &0.893                  &0.687                      &5.631\\
MDD-RKHS-5                      &\underline{0.992}        &2.030                   &\underline{54.015}  &0.214              &\underline{1.554}  &0.745                &\underline{0.804}      &\underline{0.672}          &\textbf{5.350}\\
MDD-RKHS-10                     &1.292                    &2.326                   &55.120              &0.239              &1.780              &0.773                &0.849                  &0.683                      &5.534\\

   \bottomrule
\end{tabular*}
%\vspace{-0.6cm}
\end{table*}


\begin{table*}[t]
\small
\footnotesize
%\scriptsize
%\tiny
%\renewcommand{\captionfont}{\small}
   \caption{
    \small Comparison of run time (second) amound our proposed \texttt{MDD-LS} and \texttt{MDD-RKHS} with other methods.
   %We set the numbers of our method (EPSVM) to be bold if our method outperforms all other methods (KTA, CKTA, FSM, 3-CV, 5-CV and 10-CV).
   }
   \label{tabel:time}
    %\centering
    \begin{tabular*}{\linewidth}{@{\extracolsep{-0.10cm}}lccccccccc}
    \toprule
                                &madelon                  &space\_ga               &cpusmall            &phishing           &cadata             &a8a                  &a9a                    &cod\-rna                   &YearPred                \\   \hline
RR                              &2.069                    &0.280                   &1.218               &1.526              &0.490              &2.544                &2.957                  &1.866                      &10.433 \\
DRR-5                           &1.849                    &0.224                   &0.463               &0.625              &0.363              &0.773                &0.881                  &0.736                      &3.709 \\
DRR-10                          &1.623                    &0.193                   &0.298               &0.350              &0.214              &0.401                &0.503                  &0.435                      &2.645 \\
MDD-LS-5                        &1.875                    &0.235                   &0.587               &0.664              &0.427              &1.208                &1.167                  &0.876                      &5.474 \\
MDD-LS-10                       &1.656                    &0.214                   &0.315               &0.395              &0.269              &0.651                &0.628                  &0.412                      &3.156 \\
\hline \hline
KRR                             &3.450                    &1.508                   &9.801               &12.08              &76.99              &15.33                &16.103                 &137.6                      &/ \\
KDRR-5                          &2.487                    &0.295                   &3.374               &1.451              &5.524              &6.021                &5.913                  &40.22                      &86.754\\
KDRR-10                         &1.653                    &0.183                   &1.863               &0.689              &0.302              &3.670                &3.544                  &23.64                      &46.197\\
MDD-RKHS-5                      &2.692                    &0.331                   &5.637               &1.901              &29.85              &8.628                &9.454                  &73.09                      &167.208\\
MDD-RKHS-10                     &1.781                    &0.206                   &3.024               &0.984              &17.78              &4.125                &5.679                  &40.23                      &89.312\\

   \bottomrule
\end{tabular*}
\vspace{-0.6cm}
\end{table*}

The run time is reported in Table \ref{tabel:time},
which can be summarized as follows:
1) Global methods cost more time than distributed methods on all data sets;
2) Kernel methods always spend more time than Linear methods, because of higher computation complexity;
3) Distributed methods lead great speedup on some data sets, such as space\_ga, phishing and cadata;
4) The running time of distributed methods decays almost linearly associated with the increase of worker nodes;
5) Compared with global methods, our \texttt{MDD} methods own higher computational efficiency, while existing small distance away from divide-and-conquer methods.

The above results show that \texttt{MDD} methods need a bit more training time
but make the performance gap between global methods and traditional distributed methods tighter,
which is consistent with our theoretical analysis.


\section{Conclusion}
In this paper, we studied the generalization performance of distributed learning,
and derived a sharper generalization error bound,
which is much sharper than existing  generalization bounds of divide-and-conquer based distributed learning.
Then, we designed two algorithms with statistical guarantees and fast convergence rates for linear space and RKHS:
\texttt{MDD-LS} and \texttt{MDD-RKHS}.
%a convex optimization way based on any existing MC-MKL algorithms,
%and the other that put local Rademancher complexity in optimization formulation,
%for which we give a stochastic mirror and sub-gradient descent method.
Empirical results show our methods outperform the popular divide-and-conquer method but only with little additional time.
%Based on max-discrepancy of each local estimate, our analysis can be used as a solid basis for the
%design of new distributed learning algorithms.

\bibliographystyle{abbrv}
\bibliography{NIPS2018}
\end{document}

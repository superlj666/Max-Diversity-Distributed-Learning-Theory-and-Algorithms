\documentclass{article}

% if you need to pass options to natbib, use, e.g.:
% \PassOptionsToPackage{numbers, compress}{natbib}
% before loading nips_2018

% ready for submission
\usepackage{nips_2018}

% to compile a preprint version, e.g., for submission to arXiv, add
% add the [preprint] option:
% \usepackage[preprint]{nips_2018}

% to compile a camera-ready version, add the [final] option, e.g.:
% \usepackage[final]{nips_2018}

% to avoid loading the natbib package, add option nonatbib:
% \usepackage[nonatbib]{nips_2018}

\usepackage[utf8]{inputenc} % allow utf-8 input
\usepackage[T1]{fontenc}    % use 8-bit T1 fonts
\usepackage{hyperref}       % hyperlinks
\usepackage{url}            % simple URL typesetting
\usepackage{booktabs}       % professional-quality tables
\usepackage{amsfonts}       % blackboard math symbols
\usepackage{nicefrac}       % compact symbols for 1/2, etc.
\usepackage{microtype}      % microtypography
\usepackage{bm}
\usepackage{amsmath}
\usepackage{amsthm}
\usepackage{amssymb,amsfonts}
\newcommand{\argmax}{\operatornamewithlimits{arg\,max}}
\newcommand{\argmin}{\operatornamewithlimits{arg\,min}}
\newtheorem{assumption}{Assumption}
\newtheorem{lemma}{Lemma}
\newtheorem{theorem}{Theorem}
\newtheorem{corollary}{Corollary}
\title{Distributed Learning: Risk Bound and Algorithm}

% The \author macro works with any number of authors. There are two
% commands used to separate the names and addresses of multiple
% authors: \And and \AND.
%
% Using \And between authors leaves it to LaTeX to determine where to
% break the lines. Using \AND forces a line break at that point. So,
% if LaTeX puts 3 of 4 authors names on the first line, and the last
% on the second line, try using \AND instead of \And before the third
% author name.

\author{
  %David S.~Hippocampus\thanks{Use footnote for providing further
%    information about author (webpage, alternative
%    address)---\emph{not} for acknowledging funding agencies.} \\
%  Department of Computer Science\\
%  Cranberry-Lemon University\\
%  Pittsburgh, PA 15213 \\
%  \texttt{hippo@cs.cranberry-lemon.edu} \\
  %% examples of more authors
  %% \And
  %% Coauthor \\
  %% Affiliation \\
  %% Address \\
  %% \texttt{email} \\
  %% \AND
  %% Coauthor \\
  %% Affiliation \\
  %% Address \\
  %% \texttt{email} \\
  %% \And
  %% Coauthor \\
  %% Affiliation \\
  %% Address \\
  %% \texttt{email} \\
  %% \And
  %% Coauthor \\
  %% Affiliation \\
  %% Address \\
  %% \texttt{email} \\
}

\begin{document}
% \nipsfinalcopy is no longer used

\maketitle

\begin{abstract}

\end{abstract}

\section{Preliminaries}
We consider  the supervised learning where a learning algorithm receives a sample of $N$ labeled points
\begin{align*}
  \mathcal{S}=\left\{z_i=(\mathbf  x_i,y_i)\right\}_{i=1}^N
        \in (\mathcal{Z}=\mathcal{X}\times \mathcal{Y})^N,
\end{align*}
where $\mathcal{X}$ denotes the input space and
$\mathcal{Y}$ denotes the output space.
%$\mathcal{Y}\subset \mathbb{R}$ in the regression case and
%$\mathcal{Y}=\{-1,+1\}$ in classification case.
We assume $\mathcal{S}$ is drawn identically and independently from a fixed,
but unknown probability  distribution $\mathbb{P}$ on
$\mathcal{Z}=\mathcal{X}\times\mathcal{Y}$.
The goal is to learn a good prediction model $f\in\mathcal{H}:\mathcal{X}\rightarrow\mathcal{Y}$,
whose prediction accuracy at instance
$z=(\mathbf x,y)$ is measured by a loss function $\ell(f,z)$.

In this paper, we focus on the supervised learning over the some Hilbert space $\mathcal{H}$:
\begin{align}
  \hat{f}=\argmin_{f\in\mathcal{H}} \hat{R}(f)=\frac{1}{N}\sum_{i=1}^N\ell(f,z_i)+r(f)
\end{align}
where $\ell(f,z)$ is the loss function, and $r(f)$ is a regularizer.

The expect of $\hat{R}$ is defined as
\begin{align}
  R(f)=\mathbb{E}_{z\sim \mathbb{P}}[\ell(f,z)]+ r(f).
\end{align}
Let $f^\ast=\argmin_{f\in\mathcal{H}} R(f)$,  and $R_\ast=R(f^\ast)$.

In the distributed setting, we divide evenly amongst $m$ processors or inference procedures.
Let $\mathcal{S}_i, i\in(1,2,\ldots,m)$, denote a subsampled dataset of size $n=\frac{N}{m}$.
For each $i=1,2,\ldots,m$, the local estimate
\begin{align*}
  \hat{f}_i=\argmin_{f\in\mathcal{H}}\hat{R}_i(f)=
  \left\{
    \frac{1}{n}\sum_{z_i\in\mathcal{S}_i}\ell(f,z_j)+r(f).
  \right\}
\end{align*}
The average local estimates is denote as
\begin{align*}
  \bar{f}=\frac{1}{m}\sum_{i=1}^m\hat{f}_i.
\end{align*}

In the next, we will estimate the discrepancy of $R(\bar{f})$ and $R(f^\ast)$.
\section{Faster Rates of Distributed Learning}
\subsection{Assumptions}
In the following, we use $\|\cdot\|_\mathcal{H}$ to denote the norm induced by inner product of the Hilbert space $\mathcal{H}$.
\begin{assumption}
\label{assumption-strongly-convex}
  The function $\nu(f,z)=\ell(f,z)+r(f)$ is $\eta$-strongly convex and $\beta$-smooth with respect to the first variable $f$,
  that is $\forall f,f'\in\mathcal{H}$, $z\in\mathcal{Z}$,
  \begin{align}
    \label{assumption-strongly}
     \langle \nabla \nu(f,z), f-f'\rangle_\mathcal{H}+\frac{\eta}{2}\|f-f'\|_\mathcal{H} &\leq \nu(f,z)-\nu(f',z),\\
     \label{assumption-smooth}
     \left\|\nabla\nu(f,z)-\nabla \nu(f',z)\right\|_\mathcal{H}&\leq \beta\|f-f'\|_\mathcal{H}.
  \end{align}
\end{assumption}
The above assumptions allow us to model many popular losses,
such as square loss and logistic loss, and the regularizer, $r(f)=\lambda \|f\|_\mathcal{H}^2$.
\begin{assumption}
\label{assumption-optimal-bound}
  Let $f_\ast=\argmin_{f\in\mathcal{H}}R(f)$. We assume that the gradient at $f_\ast$ is upper bounded by $M$, that is
  \begin{align*}
    \|\nabla \ell(f^\ast,z)\|_\mathcal{H}\leq M, \forall z\in\mathcal{Z}.
  \end{align*}
\end{assumption}
The above assumption is a common assumption can be seen in \cite{zhang2012communication,Zhang2017er}.
\section{Faster Rates of Distributed Learning}
\begin{theorem}
\label{theorem-main}
For any $0<\delta<1$, $\epsilon$, the cover number of the Hilbert space $\mathcal{H}$ is defined as $\mathcal{N}(\mathcal{H},\epsilon)$.
Under \textbf{Assumptions} \ref{assumption-strongly-convex} and \ref{assumption-optimal-bound},
and if
  \begin{align}
    \label{equation-12}
    m\leq \frac{N\eta}{4\beta\log\mathcal{N}(\mathcal{H},\epsilon)},
  \end{align}
  with probability at least $1-\delta$,
  we have
  \begin{align}
    \label{equation-13}
    \begin{aligned}
    R(\bar{f})-R(f_\ast)&\leq
    \frac{16\beta \log(4m/\delta)}{n^2\eta}+\frac{128\beta R_\ast\log(4m/\delta)}{n\eta}+\frac{32\beta^2\epsilon^2}{\eta}+
    \frac{64\beta L \log\left(\mathcal{N}(\mathcal{H},\epsilon)\right)\epsilon}{n\eta}\\
   &~~~~~~\frac{64\beta \log^2\left(\mathcal{N}(\mathcal{H},\epsilon)\right)\epsilon^2}{n^2\eta}
   -\Delta(\bar{f}),
  \end{aligned}
  \end{align}
  where $R_\ast=R(f^\ast)$, $\Delta_{\bar{f}}=\frac{\eta}{4m^2}\sum_{i,j=1,i\not=j}^m\|\hat{f}_i-\hat{f}_j\|_\mathcal{H}^2$
\end{theorem}
By choosing $\epsilon$ small enough,
\begin{align*}
\frac{32\beta^2\epsilon^2}{\eta}+
    \frac{64\beta L \log\left(\mathcal{N}(\mathcal{H},\epsilon)\right)\epsilon}{n\eta}
    +\frac{64\beta \log^2\left(\mathcal{N}(\mathcal{H},\epsilon)\right)\epsilon^2}{n^2\eta}
\end{align*}
will becomes non-dominating.
To be specific, we have the following corollary:
\begin{corollary}
\label{corollary-first}
  By setting $\epsilon=\frac{1}{n}$ in Theorem \ref{theorem-main},
  with high probability,
  we have
  \begin{align*}
    R(\bar{f})-R(f_\ast)=\mathcal{O}\left(\frac{R_\ast\log(m)}{n}
    +\frac{\log(\mathcal{N}(\mathcal{H},\frac{1}{n}))}{n^2}
    -\Delta(\bar{f})\right).
  \end{align*}
\end{corollary}

\subsection{Linear Hypothesis Space}
\label{subsection-3.1}
If we consider the linear hypothesis space, that is
\begin{align*}
\mathcal{H}=\left\{f=\mathbf w^\mathrm T\mathbf x|\mathbf w\in \mathbb{R}^d, \|\mathbf w\|_2\leq B\right\}.
\end{align*}
According to the \cite{pisier1999volume},
the cover number of linear hypothesis space can be bounded:
\begin{align*}
  \log\left(\mathcal{N}(\mathcal{H},1/n)\right)\leq d\log \left(6Bn\right).
\end{align*}
Thus, from Corollary \ref{corollary-first}, we have
\begin{align*}
  R(\bar{f})-R(f_\ast)&=\mathcal{O}\left(\frac{R_\ast\log m}{n}+\frac{d\log n}{n^2}-
  \Delta(\bar{f})\right)
  %&\leq \mathcal{O}\left(\frac{1}{n^2}+\frac{R_\ast\log m}{n}+\frac{d\log n}{n^2}\right)
\end{align*}
When the minimal risk is small, i.e., $R_\ast=\mathcal{O}\left(\frac{d}{n}\right)$,
the rate is improved to
\begin{align*}
    \mathcal{O}\left(\frac{d\log (mn)}{n^2}-\Delta(\bar{f})\right)=\mathcal{O}\left(\frac{d\log N}{n^2}-\Delta(\bar{f})\right).
\end{align*}
Therefore, if $m\leq \sqrt{\frac{N}{d\log N}}$, we have
\begin{align*}
    R(\bar{f})-R(f_\ast)=\mathcal{O}\left(\frac{1}{N}-\Delta(\bar{f})\right).
\end{align*}
\subsection{Reproducing Kernel Hilbert Space}
\label{subsection-3.2}
The reproducing kernel Hilbert space $\mathcal{H}_K$ associated with the kernel $K$ is
defined to be the closure of the linear span of the set of functions
$\left\{K(\mathbf x,\cdot):\mathbf x\in\mathcal{X}\right\}$ with the inner product satisfying
\begin{align*}
  \langle K(\mathbf x,\cdot), f\rangle_{\mathcal{H}_K}=f(\mathbf x), \forall \mathbf x\in\mathcal{X}, f\in\mathcal{H}_K.
\end{align*}

In this subsection, we consider hypothesis space as the reproducing kernel Hilbert space,
\begin{align*}
  \mathcal{H}:=\left\{f\in\mathcal{H}_{K}: \|f\|_{\mathcal{H}_K}\leq B\right\}.
\end{align*}

From \cite{zhou2002covering}, if the Mercer kernel
\begin{align*}
  K(\mathbf x,\mathbf x')=k(\mathbf x-\mathbf x'), k(\mathbf x)=\exp\left\{-\frac{\|\mathbf x\|^2}{\sigma^2}, \mathbf x,\mathbf x' \in[0,1]^d, \right\},
\end{align*}
then for $0\leq \epsilon\leq B/2$, there holds:
\begin{align*}
 \log \left(\mathcal{N}(\mathcal{H},1/n)\right)=\mathcal{O}\left(\log^d(nB)\right)
\end{align*}
According to Corollary \ref{corollary-first}, we can obtain that
\begin{align*}
  R(\bar{f})-R(f_\ast)=\mathcal{O}\left(\frac{R_\ast\log m}{n}+\frac{\log^d n}{n^2}-
  \Delta(\bar{f})\right).
  %&\leq \mathcal{O}\left(\frac{1}{n^2}+\frac{R_\ast\log m}{n}+\frac{d\log n}{n^2}\right)
\end{align*}
When  the minimal risk  $R_\ast$ is small, $R_\ast=\mathcal{O}\left(\frac{\log^{(d-1)}n}{n}\right)$,
if $m\leq n$, we have
\begin{align*}
  R(\bar{f})-R(f_\ast)=\mathcal{O}\left(\frac{\log^d n}{n^2}-
  \Delta(\bar{f})\right)
\end{align*}
Therefore, if $m\leq \sqrt{\frac{N}{\log^2 n}}$,
we have
\begin{align*}
     R(f)-R(f_\ast)=\mathcal{O}\left(\frac{1}{N}-\Delta(\bar{f})\right)
\end{align*}
%Note that for our bound,
%\begin{align*}
%     R(f)-R(f_\ast)=\left\{
%     \begin{aligned}
%     &\mathcal{O}\left(\frac{d\log N}{n^2}-\Delta(\bar{f})\right)  &&\text{linear hypothesis space}\\
%     &\mathcal{O}\left(\frac{\log^d n}{n^2}-\Delta(\bar{f})\right) &&\text{RKHS  space}
%     \end{aligned}
%     \right.
%\end{align*}
%So, we can obtain that
%\begin{align*}
%     R(f)-R(f_\ast)=\mathcal{O}\left(\frac{1}{N}-\Delta(\bar{f})\right)
%     % \left\{
%%     \begin{aligned}
%%     &\mathcal{O}\left(\frac{1}{N}-\Delta(\bar{f})\right)  &\text{if} m\leq \sqrt{\frac{N}{d\log N}} &\text{for linear hypothesis space}\\
%%     &\mathcal{O}\left(\frac{\log^d n}{n^2}-\Delta(\bar{f})\right) &ff &\text{RKHS  space}
%%     \end{aligned}
%%     \right.
%\end{align*}
%if $m\leq \sqrt{\frac{N}{d\log N}}$ for linear hypothesis space, or $m\leq \sqrt{\frac{N}{\log^2 n}}$ for RKHS space.

\subsection{Comparison with Related Work}
Under the smooth, strongly convex and other some common assumption,
\cite{zhang2012communication} shows that
\begin{align}
  \mathbb{E}\left[\|\bar{f}-f_\ast\|^2\right]=\mathcal{O}\left(\frac{1}{N}+\frac{\log d}{n^2}\right).
\end{align}
If $\nu(f,z)$ is $L$-Lipschitz continuous over $f$, that is
\begin{align*}
  \forall f, f\in \mathcal{H}, z\in\mathcal{Z}, |\nu(f,z)-\nu(f',z)|\leq L\|f-f'\|_\mathcal{H},
\end{align*}
it is easy to verity that
\begin{align}
  \nonumber R(f)-R(f_\ast)&\leq L\mathbb{E}\left[\|\bar{f}-f_\ast\|_\mathcal{H}\right]\leq L\sqrt{\mathbb{E}\left[\|\bar{f}-f_\ast\|^2_\mathcal{H}\right]}\\
  \label{related-work-one}    &= \mathcal{O}\left(\frac{1}{\sqrt{N}}+\frac{\sqrt{\log d}}{n}\right).
\end{align}
According to the subsections \label{subsection-3.1} and \label{subsection-3.2},
we know that if $m$ is not very large,
the order of this paper can reach $\mathcal{O}\left(\frac{1}{N}-\Delta(\bar{f})\right)$,
which is much sharper than the order of \eqref{related-work-one}.

\cite{Zhang2013} consider the kernel ridge regression, under some assumptions over the feature map induced by the kernel function,
and if $m$ is not very large,
they show that
\begin{align*}
   \mathbb{E}\left[\|\bar{f}-f_\ast\|^2\right]=\mathcal{O}\left(\frac{1}{N}\right).
\end{align*}
If $\nu(f,z)$ is $L$-Lipschitz continuous over $f$,
same as the above analysis,
it is easy to verity that
\begin{align*}
  R(f)-R(f_\ast)=\mathcal{O}\left(\frac{1}{\sqrt{N}}\right),
\end{align*}
which is much looser than of propose bound.
\section{Discrepant Distributed Algorithms (DDA)}
According to the above results, under some assumptions, we know that
  \begin{align*}
     R(f)-R(f_\ast)=\mathcal{O}\left(\frac{1}{N}-\Delta(\bar{f})\right),
\end{align*}
where $\Delta_{\bar{f}}=\frac{\eta}{4m^2}\sum_{i,j=1,i\not=j}^m\|\hat{f}_i-\hat{f}_j\|_\mathcal{H}^2$.
Thus, to obtain tight bound, the discrepancy of each local estimate $\hat{f}_i, i=1,\ldots,m$ should be large.
Therefore, it is reasonable to derive the following optimization problem:
\begin{align}
  \label{opt-problem}
  \hat{f}_i=\argmin_{f\in\mathcal{H}}\frac{1}{n}\sum_{z_j\in\mathcal{S}_i}\ell(f,z_i)+r(f)-\gamma \|f-\bar{f}_{\backslash i}\|_\mathcal{H},
\end{align}
where $\bar{f}_{\backslash i}=\frac{1}{m-1}\sum_{j=1,j\not =i}^m\hat{f}_j$.
\subsection{Linear Hypothesis Space}
When $\mathcal{H}$ is a linear Hypothesis space,
we consider the following problem:
\begin{align*}
  \hat{\mathbf w}_i=\argmin_{\mathbf w\in\mathbb{R}^d}
  \frac{1}{n}\sum_{z_i\in\mathcal{S}_i}(\mathbf w^\mathrm{T}\mathbf x_i-y_i)^2+\lambda \|\mathbf w\|_2^2-\gamma \|\mathbf w-\bar{\mathbf w}_{\backslash i}\|_2^2,
\end{align*}
where $\bar{\mathbf w}_{\backslash i}=\frac{1}{m-1}\sum_{j=1,j\not =i}\hat{\mathbf w}_j$.
If given $\bar{\mathbf w}_{\backslash i}=\frac{1}{m-1}\sum_{j=1,j\not =i}\hat{\mathbf w}_j$,
it is easy to verity that $\hat{\mathbf w}_i$ can be written as
\begin{align*}
  \hat{\mathbf w}_i=\left(\frac{1}{n}\mathbf X_{\mathcal{S}_i}\mathbf X_{\mathcal{S}_i}^\mathrm{T}+\lambda \mathbf I_d-\gamma \mathbf I_d\right)^{-1}
  \left(\frac{1}{n}\mathbf X_{\mathcal{S}_i}\mathbf y_{\mathcal{S}_i}-\gamma \bar{\mathbf w}_{\backslash i}\right),
\end{align*}
where $\mathbf X_{\mathcal{S}_i}=(\mathbf x_{t_1},\mathbf x_{t_2},\ldots, \mathbf x_{t_n})$,
$\mathbf y_{\mathcal{S}_i}=(y_{t_1},y_{t_2},\ldots,y_{t_n})^\mathrm{T}$, $z_{t_i}\in \mathcal{S}_i$, $i=1,\ldots, n$.


Let $\mathbf A_i=\frac{1}{n}\mathbf X_{\mathcal{S}_i}\mathbf X_{\mathcal{S}_i}^\mathrm{T}+\lambda \mathbf I_d-\gamma \mathbf I_d$,
$\mathbf b_i=\frac{1}{n}\mathbf X_{\mathcal{S}_i}^\mathrm{T}\mathbf y_{\mathcal{S}_i}$.

Let $\mathbf d_i=\mathbf A_i^{-1}\bar{\mathbf w}_{\backslash i}$,
  $\hat{\mathbf w}_i=\mathbf A_i^{-1} \mathbf b_i$,
  we have
  \begin{align*}
    \bar{\mathbf w}_{\backslash i}^\mathrm{T}\hat{\mathbf w}_i=\bar{\mathbf w}_{\backslash i}^\mathrm{T}\mathbf A_i^{-1} \mathbf b_i
    =\left(\mathbf A_i^{-1}\bar{\mathbf w}_{\backslash i}\right)^\mathrm{T}\mathbf b_i=\mathbf d_i^\mathrm{T}\mathbf b_i,
  \end{align*}
  thus $\mathbf d_i=\frac{\bar{\mathbf w}_{\backslash i}^\mathrm{T}\hat{\mathbf w}_i}{\mathbf b_i}$

The DDA Algorithm is given as follows:
\begin{itemize}
  \item[\textbf{Input}]: $\lambda,\gamma$, $\mathbf X$, $m$, $\zeta>0$.
  %\item[1] For branch node $i$: \\
%   \qquad Compute $\hat{\mathbf w}_i^0=\mathbf A_i^{-1} \mathbf b_i$, push $\hat{\mathbf w}_i^0$ to center node;\\
%           Center node: $\bar{\mathbf w}^0_{\backslash i}=\frac{1}{m-1}\sum_{j=1,j\not=i}\hat{\mathbf w}_i^0$
  %$i=1,\ldots, m$,
%        where $\mathbf A_i=\left(\frac{1}{n}\mathbf X_{\mathcal{S}_i}\mathbf X_{\mathcal{S}_i}^\mathrm{T}
 % +\lambda \mathbf I_n+\gamma \mathbf I_n\right)^{-1}$,
  %$\mathbf b_i=\frac{1}{n}\mathbf X_{\mathcal{S}_i}^\mathrm{T}\mathbf y_{\mathcal{S}_i}$,
%  $\hat{\mathbf w}_i^0=\mathbf A_i^{-1} \mathbf b_i$, $i=1,\ldots, m$,
  %\item[1]: Singular value decomposition:
%  $\frac{1}{n}\mathbf X_{\mathcal{S}_i}\mathbf X_{\mathcal{S}_i}^\mathrm{T}=\mathbf V_{\mathcal{S}_i}
%  \mathbf \Sigma_{\mathcal{S}_i}\mathbf V^\mathrm{T}_{\mathcal{S}_i}$,
%  $i=1,\ldots,m$.\\
%  $\hat{\mathbf w}_i^0=\mathbf V_{\mathcal{S}_i}\left(\mathbf \Sigma_{\mathcal{S}_i}+(\lambda+\gamma)\mathbf I_n\right)^{-1}
%  \mathbf V^\mathrm{T}_{\mathcal{S}_i}\left(\frac{1}{n}\mathbf X_{\mathcal{S}_i}^\mathrm{T}\mathbf y_{\mathcal{S}_i}\right)$,$i=1,\ldots,m$\\
%  $\bar{\mathbf w}_{\backslash i}^0=\frac{1}{m-1}\sum_{j=1,j\not=i}\hat{\mathbf w}_i^0$, $i=1,\ldots, m$
  \item[\textbf{For}] $t=0, 1, \ldots, T$\\
  %\textbf{If} $t=0$
%  \begin{itemize}
%    \item For branch node $i$: $\hat{\mathbf w}_i^0=\mathbf A_i^{-1} \mathbf b_i$, push $\hat{\mathbf w}_i^0$ to center node;
%    \item Center node: $\bar{\mathbf w}^0_{\backslash i}=\frac{1}{m-1}\sum_{j=1,j\not=i}\hat{\mathbf w}_i^0$
%  \end{itemize}
  Each branch node $i$:
   \begin{itemize}
    \item[\textbf{If}]$t=0$\\
     $\hat{\mathbf w}_i^0=\mathbf A_i^{-1} \mathbf b_i$;
    \end{itemize}
    %\item[]
    \begin{itemize}
    \item[\textbf{else}]
    \item[ ]
     $\mathbf d_i^t=\frac{\left(\bar{\mathbf w}^0_{\backslash i}\right)^\mathrm{T}\hat{\mathbf w}_i^0}{\mathbf b_i}$\\
    $\hat{\mathbf w}_i^t=\hat{\mathbf w}_i^0-\gamma\mathbf d_i^t$;\\
    %$\bar{\mathbf w}^{t}_{\backslash i}=\frac{1}{m-1}\sum_{j=1,j\not=i}\hat{\mathbf w}_i^t$;\\
     push $\hat{\mathbf w}_i^t$ to center node;
    \end{itemize}
   Center node:
    \begin{itemize}
      \item[ ]
      $\bar{\mathbf w}^t=\frac{1}{m}\sum_{i=1}^m\hat{\mathbf w}_i^t$\\
      \textbf{If} $\|\bar{\mathbf w}^{t}-\bar{\mathbf w}^{t-1}\|\leq \zeta$, \textbf{End}\\
              \textbf{else} push $\bar{\mathbf w}^{t}_{\backslash i}=\frac{m\bar{\mathbf w}^{t}-\hat{\mathbf w}_i^t}{m-1}$ to each branch node $i$
    \end{itemize}
    \item[\textbf{End}]
    \item[\textbf{Output}]: $\bar{\mathbf w}=\frac{1}{m}\sum_{i=1}^m\hat{\mathbf w}_i^T$
\end{itemize}
\subsection{Reproducing Kernel Hilbert Space}
When $\mathcal{H}$ is a reproducing kernel Hilbert space,
we consider the following problem:
\begin{align}
\label{opti-RKHS-first}
  \hat{f}_i=
  \argmin_{f\in\mathcal{H}}
  \frac{1}{n}\sum_{z_i\in\mathcal{S}_i}(f(\mathbf x_i)-y_i)^2+\lambda \|f\|_\mathcal{H}^2+\gamma \|f-\bar{f}_{\backslash i}\|_\mathcal{H}^2,
  %\frac{1}{n}\sum_{z_k\in\mathcal{S}_i}\left(\sum_{j=1}^n c_j K(\mathbf x_j,\mathbf x_k)-y_k\right)^2+
%  \lambda \left\|\sum_{j=1}^n c_j K(\mathbf x_j,\cdot)\right\|_\mathcal{H}^2+\gamma \mathbf w-\bar{\mathbf w}_{\backslash i}\|_2^2,
\end{align}
where $f(\mathbf x)=\sum_{j=1}^n c_j K(\mathbf x_j,\mathbf x)$,
which can be written as
\begin{align}
\label{opti-RKHS}
  \hat{\mathbf c}_i=\argmin_{\mathbf c\in\mathbb{R}^n}
  \frac{1}{n}\|\mathbf K_{\mathcal{S}_i}\mathbf c-\mathbf y_{\mathcal{S}_i}\|_2^2+\lambda \mathbf c^\mathrm{T}\mathbf K_{\mathcal{S}_i}\mathbf c
  -\gamma \left(\mathbf c- \bar{\mathbf c}_{\backslash i}\right)^\mathrm{T}
  \mathbf K_{\mathcal{S}_i}\left(\mathbf c- \bar{\mathbf c}_{\backslash i}\right).
\end{align}
If given $\bar{\mathbf c}_{\backslash i}=\frac{1}{m-1}\sum_{j=1,j\not =i}\hat{\mathbf c}_j$,
it is easy to verity that $\hat{\mathbf c}_i$ can be written as
\begin{align*}
  \hat{\mathbf c}_i=\left(\mathbf K_{\mathcal{S}_i}+\lambda \mathbf I_n-\gamma\mathbf I_n\right)^{-1}
  \left(\mathbf y_{\mathcal{S}_i}-\gamma \bar{\mathbf c}_{\backslash i}\right)
\end{align*}
Let $\mathbf A_i=\mathbf K_{\mathcal{S}_i}+\lambda \mathbf I_n-\gamma\mathbf I_n$, $\mathbf b_i=\mathbf y_{\mathcal{S}_i}$.
\begin{itemize}
  \item[\textbf{Input}]: $\lambda,\gamma$, $\mathbf X$, $m$, $\zeta>0$.
  \item[\textbf{For}] $t=0, 1, \ldots, T$\\
  %\textbf{If} $t=0$
%  \begin{itemize}
%    \item For branch node $i$: $\hat{\mathbf w}_i^0=\mathbf A_i^{-1} \mathbf b_i$, push $\hat{\mathbf w}_i^0$ to center node;
%    \item Center node: $\bar{\mathbf w}^0_{\backslash i}=\frac{1}{m-1}\sum_{j=1,j\not=i}\hat{\mathbf w}_i^0$
%  \end{itemize}
  Each branch node $i$:
   \begin{itemize}
    \item[\textbf{If}]$t=0$\\
     $\hat{\mathbf c}_i^0=\mathbf A_i^{-1} \mathbf b_i$;
    \end{itemize}
    %\item[]
    \begin{itemize}
    \item[\textbf{else}]
    \item[ ]
     $\mathbf d_i^t=\frac{\left(\bar{\mathbf c}^0_{\backslash i}\right)^\mathrm{T}\hat{\mathbf c}_i^0}{\mathbf b_i}$\\
    $\hat{\mathbf c}_i^t=\hat{\mathbf c}_i^0-\gamma\mathbf d_i^t$;\\
    %$\hat{\mathbf c}_i^t=\hat{\mathbf c}_i^0-\gamma\mathbf A_i^{-1}\bar{\mathbf c}^{t-1}_{\backslash i}$;\\
    %$\bar{\mathbf w}^{t}_{\backslash i}=\frac{1}{m-1}\sum_{j=1,j\not=i}\hat{\mathbf w}_i^t$;\\
     push $\hat{\mathbf w}_i^t$ to center node;
    \end{itemize}
   Center node:
    \begin{itemize}
      \item[ ]
      $\bar{\mathbf c}^t=\frac{1}{m}\sum_{i=1}^m\hat{\mathbf c}_i^t$\\
      \textbf{If} $\|\bar{\mathbf c}^{t}-\bar{\mathbf c}^{t-1}\|\leq \zeta$, \textbf{End}\\
              \textbf{else} push $\bar{\mathbf c}^{t}_{\backslash i}=\frac{m\bar{\mathbf c}^{t}-\hat{\mathbf c}_i^t}{m-1}$ to each branch node $i$
    \end{itemize}
    \item[\textbf{End}]
    \item[\textbf{Output}]: $\bar{\mathbf c}=\frac{1}{m}\sum_{i=1}^m\hat{\mathbf c}_i^T$
\end{itemize}

\section{Analysis}
\subsection{The Key Idea}
Since $R(f)$ is $\eta$-strongly convex function,
we have
\begin{align*}
  R(\bar{f})\leq \frac{1}{m}\sum_{i=1}^mR(\hat{f}_i)-\frac{\eta}{2m^2}\sum_{i,j=1}^m\|\hat{f}_i-\hat{f}_j\|^2.
\end{align*}
Therefore, we have
\begin{align}
\label{equaiton-strongly-ff}
  R(\bar{f})-R(f_\ast)\leq \frac{1}{m}\sum_{i=1}^m\left[R(\hat{f}_i)-R(f_\ast)\right]
  -\frac{\eta}{4m^2}\sum_{i,j=1,i\not=j}^m\|\hat{f}_i-\hat{f}_j\|^2.
\end{align}
In the next, we will estimate $R(\hat{f}_i)-R(f_\ast)$.

Our theoretical analysis is built upon the following inequality:
\begin{align}
 \nonumber &~~~~~R(\hat{f}_i)-R(f_\ast)+\frac{\eta}{2}\|\hat{f}_i-f_\ast\|^2
  \leq \langle \nabla R(\hat{f}_i),\hat{f}_i-f_\ast\rangle
  \\ \nonumber&=\langle \nabla R(\hat{f}_i)-\nabla R(f_\ast)-[\nabla \hat{R}(\hat{f}_i)-\nabla \hat{R}(f_\ast)],
   \hat{f}_i-f_\ast\rangle+\langle
   \nabla \hat{R}(\hat{f}_i)-\nabla \hat{R}(f_\ast)+\nabla R(f_\ast), \hat{f}_i-f_\ast\rangle\\
   &\nonumber=\langle \nabla R(\hat{f}_i)-\nabla R(f_\ast)-[\nabla \hat{R}(\hat{f}_i)-\nabla \hat{R}(f_\ast)],
   \hat{f}_i-f_\ast\rangle+\langle \nabla R(f_\ast)-\nabla \hat{R}(f_\ast), \hat{f}_i-f_\ast\rangle\\
   \label{equation-important-middle}
   &\leq \left(\underbrace{\left\|\nabla R(\hat{f}_i)-\nabla R(f_\ast)-[\nabla \hat{R}(\hat{f}_i)-\nabla \hat{R}(f_\ast)]\right\|}_{:=A_1}+
   \underbrace{\left\|\nabla R(f_\ast)-\nabla \hat{R}(f_\ast)\right\|}_{:A_2}\right)\left\|\hat{f}_i-f_\ast \right\|
\end{align}
where $\eta$ is the strong convexity modulus of $R(\cdot)$ if exists otherwise it is zero.
\begin{lemma}
  \label{lem-nonequation}
    Let $\mathcal{H}$ be a Hilbert space and let $\xi$ be a random variable with values in $\mathcal{H}$.
    Assume $\|\xi\|\leq M\leq \infty$ almost surely.
    Denote $\sigma^2(\xi)=\mathbb{E}[\|\xi\|^2]$.
    Let $\{\xi_i\}_{i=1}^n$ be $m$ independent drawers of $\xi$.
    For any $0\leq \delta\leq 1$, with confidence $1-\delta$,
    \begin{align*}
      \left\|\frac{1}{n}\sum_{j=1}^n[\xi_j-\mathbb{E}[\xi_j]]\right\|\leq \frac{2M\log(2/\delta)}{n}+\sqrt{\frac{2\sigma^2(\xi)\log(2/\delta)}{n}}.
    \end{align*}
  \end{lemma}
\begin{lemma}
\label{lemma-nablaR-nablahat}
  Under \textbf{Assumptions} \ref{assumption-loss-convex}, \ref{assumption-loss-bound} and \ref{assumption-regularizer-lipschitz}, with probability at least $1-\delta$,
  for any $f\in\mathcal{N}(\mathcal{H},\epsilon)$,
  we have
  \begin{align}
  \label{equation-nabalRR-empRR}
  \begin{aligned}
 % \nonumber
    &~~~~\left\|\nabla R(f)-\nabla R(f_\ast)-[\nabla \hat{R}(f)-\nabla \hat{R}(f_\ast)]\right\| \\
   &\leq \frac{\beta \log\left(\mathcal{N}(\mathcal{H},\epsilon)\right)\|f-f_\ast\|}{n}+\sqrt{\frac{\beta \log\left(\mathcal{N}(\mathcal{H},\epsilon)\right)(R(f)-R(f_\ast))}{n}}.
  \end{aligned}
  \end{align}
\end{lemma}
\begin{proof}
  Let $\nu(f)=\ell(f,\cdot)+r(f)$.
  Note that $\nu(f)$ is $\beta$-smooth,
  so we have
  \begin{align}
    \|\nabla\nu(f)-\nabla\nu(f_\ast)\| \leq \beta\|f-f_\ast\|
  \end{align}
  Because $\nu$ is $\beta$-smooth and convex, by (2.1.7) of?,
  we have
  \begin{align*}
   \left\|\nabla\nu (f)-\nabla \nu (f_\ast)\right\|^2
    \leq
    \beta\left(\nu (f)-\nu (f_\ast)
    -\langle \nabla\nu (f_\ast), f-f_\ast\rangle \right).
  \end{align*}
  Taking expectation over both sides, we have
  \begin{align*}
    &~~~~\mathbb{E}[\left\|\nabla\nu (f)-\nabla\nu (f_\ast)\right\|^2]\\
    &\leq \beta \left(R(\hat{f}_i)-R(f_\ast)-\langle \nabla R(f_\ast), f-f_\ast\rangle\right)\\
    &\leq \beta \left(R(\hat{f}_i)-R(f_\ast)\right)
  \end{align*}
  where the last inequality follows from the optimality condition of $f_\ast$, i.e.,
  \begin{align*}
    \langle \nabla R(f_\ast),f-f_\ast\rangle \geq 0,\forall f\in\mathcal{H}.
  \end{align*}
\end{proof}
Following Lemma \ref{lem-nonequation}, with probability at least $1-\delta$, we have
\begin{align*}
  &~~~~\left\|
  \nabla R(f)-\nabla R(f_\ast)-[\nabla \hat{R}(f)-\nabla \hat{R}(f_\ast)]
  \right\|\\
  &=\left\|
    \nabla R(f)-\nabla R(f_\ast)
    -\frac{1}{n}\sum_{z_i\in \mathcal{S}_i}
    \left[\nabla \nu(f)-\nabla \nu(f_\ast)\right]
  \right\|\\
  &\leq \frac{2\beta\|f-f_\ast\|\log(2/\delta)}{n}+\sqrt{\frac{2\beta(R(f)-R(f_\ast))\log (2/\delta)}{n}}
\end{align*}

We obtain Lemma \ref{lemma-nablaR-nablahat} by taking the union bound over all $f\in\mathcal{N}(\mathcal{H},\epsilon)$.


\begin{lemma}
  with probability at least $1-\delta$, we have
  \begin{align}
  \label{equation-nablaR-nablaempR}
    \left\|\nabla R(f_\ast)-\nabla \hat{R}(f_\ast)\right\|\leq \frac{2M\log(2/\delta)}{n}+\sqrt{\frac{8\beta R_\ast\log(2/\delta)}{n}}.
  \end{align}
\end{lemma}
\begin{proof}
Let $\nu(f,z_i)=\ell(f,z_i)+r(f)$
  Since $\nu(\cdot,z_i)$ is $\beta$-smooth and nonegative,
  from Lemma 4 of Srebro et al. (2010), we have
  \begin{align*}
    \left\|\nabla \nu(f_\ast,z_i)\right\|^2\leq 4\beta \nu(f_\ast,z_i)
  \end{align*}
  and thus
    \begin{align*}
      \mathbb{E}_{z\sim\mathbb{P}}\left[\left\|\nabla \nu(f_\ast,z)\right\|^2\right]\leq 4\beta\mathbb{E}_{z\sim\mathbb{P}}[\nu(f_\ast,z)]=
      4\beta R(f_\ast).
    \end{align*}
    From the \textbf{Assumption}, we have $\nabla \|\nu(f_\ast, z)\|\leq M$, $\forall z\in\mathcal{Z}$.
    Then, according to Lemma \ref{lem-nonequation}, with probability at least $1-\delta$, we have
    \begin{align*}
      &\left\|\nabla R(f_\ast)-\nabla \hat{R}(f_\ast)\right\|=\left\|\nabla R(f_\ast)-\frac{1}{n}\sum_{z_j\in\mathcal{S}_i}\nabla \nu(f_\ast,z_j)\right\|\\
      &\leq \frac{2\beta \log(2/\delta)}{n}+\sqrt{\frac{8\beta R_\ast \log(2/\delta)}{n}}.
    \end{align*}
\end{proof}

\begin{theorem}
  At least $1-2\delta$,
  if
  \begin{align}
    \label{equation-12}
    n\geq \frac{4\beta\log\left(\mathcal{N}(\mathcal{H},\epsilon)\right)}{\eta},
  \end{align}
  we have
  \begin{align}
    \label{equation-13}
    \begin{aligned}
    R(\bar{f})-R(f_\ast)&\leq
    \frac{16\beta \log(2m/\delta)}{n^2\eta}+\frac{128\beta R_\ast\log(2m/\delta)}{n\eta}+\frac{32\beta^2\epsilon^2}{\eta}\\
   &~~~~+
   \frac{64\beta L \log\left(\mathcal{N}(\mathcal{H},\epsilon)\right)\epsilon}{n\eta}+
   \frac{64\beta \log^2\left(\mathcal{N}(\mathcal{H},\epsilon)\right)\epsilon^2}{n^2\eta}\\
  &~~~~-\frac{\eta}{4m^2}\sum_{i,j=1,i\not=j}^m\|\hat{f}_i-\hat{f}_j\|^2.
  \end{aligned}
  \end{align}
\end{theorem}
\begin{proof}
From the property of $\epsilon$-net, we know that there exists a point
$\tilde{f}\in\mathcal{N}(\mathcal{H},\epsilon)$ such that $$\|\hat{f}_i-\tilde{f}\|\leq \epsilon.$$
According to \textbf{Assumption} \ref{assumption-loss-bound-smooth}, we have
\begin{align}
  \nonumber
    &~~~~~\left\|\nabla R(\hat{f}_i)-\nabla R(f_\ast)-[\nabla \hat{R}(\hat{f}_i)-\nabla \hat{R}(f_\ast)]\right\| \\
   \nonumber &\leq \left\|\nabla R(\tilde{f})-\nabla R(f_\ast)-[\nabla \hat{R}(\tilde{f})-\nabla \hat{R}(f_\ast)]\right\|+2\beta \epsilon\\
   \nonumber &\overset{\eqref{equation-nabalRR-empRR}}{\leq}
   \frac{\beta \log\left(\mathcal{N}(\mathcal{H},\epsilon)\right)\|\tilde{f}-f_\ast\|}{n}+
   \sqrt{\frac{\beta \log\left(\mathcal{N}(\mathcal{H},\epsilon)\right)(R(\tilde{f})-R(f_\ast))}{n}}+2\beta\epsilon\\
   \nonumber&\leq \frac{\beta \log\left(\mathcal{N}(\mathcal{H},\epsilon)\right)\|\hat{f}_i-f_\ast\|}{n}
   +\frac{\beta \log\left(\mathcal{N}(\mathcal{H},\epsilon)\right)\epsilon}{n}+2\beta\epsilon\\
   \nonumber&~~~~~+\sqrt{\frac{\beta \log\left(\mathcal{N}(\mathcal{H},\epsilon)\right)(R(\hat{f}_i)-R(f_\ast))}{n}}+
   \sqrt{\frac{\beta \log\left(\mathcal{N}(\mathcal{H},\epsilon)\right)\left(\left|R(\hat{f}_i)-R(\tilde{f})\right|\right)}{n}}\\
   \nonumber &\overset{\eqref{equation-regularizer-lipschitz}}{\leq} \frac{\beta \log\left(\mathcal{N}(\mathcal{H},\epsilon)\right)\|\hat{f}_i-f_\ast\|}{n}+
   \frac{\beta \log\left(\mathcal{N}(\mathcal{H},\epsilon)\right)\epsilon}{n}+2\beta\epsilon\\
   \label{equation-31}&~~~~+\sqrt{\frac{\beta \log\left(\mathcal{N}(\mathcal{H},\epsilon)\right)(R(\hat{f}_i)-R(f_\ast))}{n}}+
   \sqrt{\frac{\beta L \log\left(\mathcal{N}(\mathcal{H},\epsilon)\right)\epsilon}{n}}
\end{align}

  Substituting \eqref{equation-31} and \eqref{equation-nablaR-nablaempR} into \eqref{equation-important-middle},
  with probability at least $1-2\delta$, we have
  \begin{align}
    \label{equation-first-equation}
    \begin{aligned}
      &R(\hat{f}_i)-R(f_\ast)+\frac{\eta}{2}\|\hat{f}_i-f_\ast\|^2\\
      &\leq \frac{\beta \log\left(\mathcal{N}(\mathcal{H},\epsilon)\right)\|\hat{f}_i-f_\ast\|^2}{n}+
   \frac{\beta \log\left(\mathcal{N}(\mathcal{H},\epsilon)\right)\epsilon \|\hat{f}_i-f_\ast\|}{n}+2\beta\epsilon \|\hat{f}_i-f_\ast\|\\
   &~~~~+\|\hat{f}_i-f_\ast\|\sqrt{\frac{\beta \log\left(\mathcal{N}(\mathcal{H},\epsilon)\right)(R(\hat{f}_i)-R(f_\ast))}{n}}+
   \|\hat{f}_i-f_\ast\|\sqrt{\frac{\beta L \log\left(\mathcal{N}(\mathcal{H},\epsilon)\right)\epsilon}{n}}\\
   &~~~~+\frac{2\beta \log(2/\delta)\|\hat{f}_i-f_\ast\|}{n}+\|\hat{f}_i-f_\ast\|\sqrt{\frac{8\beta R_\ast \log(2/\delta)}{n}}.
    \end{aligned}
  \end{align}
  Note that
  \begin{align*}
    \sqrt{ab}\leq \frac{a}{2\eta}+\frac{b\eta}{2}, \forall a,b,\eta\geq 0.
  \end{align*}
  Therefore, we can obtain that
  \begin{align*}
  %\label{equation-finally}
%  \begin{aligned}
    \|\hat{f}_i-f_\ast\|\sqrt{\frac{\beta \log\left(\mathcal{N}(\mathcal{H},\epsilon)\right)(R(\hat{f}_i)-R(f_\ast))}{n}}
    &\leq \frac{2\beta \log\left(\mathcal{N}(\mathcal{H},\epsilon)\right)(R(\hat{f}_i)-R(f_\ast))}{n\eta}+\frac{\eta }{8}\|\hat{f}_i-f_\ast\|^2,\\
    \frac{2\beta \log(2/\delta)\|\hat{f}_i-f_\ast\|}{n}&\leq \frac{8\beta \log(2/\delta)}{n^2\eta}+\frac{\eta }{16}\|\hat{f}_i-f_\ast\|^2,\\
    \|\hat{f}_i-f_\ast\|\sqrt{\frac{8\beta R_\ast \log(2/\delta)}{n}}&\leq \frac{64\beta R_\ast\log(2/\delta)}{n\eta}+\frac{\eta }{32}\|\hat{f}_i-f_\ast\|^2,\\
    2\beta\epsilon \|\hat{f}_i-f_\ast\|&\leq \frac{32\beta^2\epsilon^2}{\eta}+\frac{\eta }{64}\|\hat{f}_i-f_\ast\|^2,\\
    \|\hat{f}_i-f_\ast\|\sqrt{\frac{\beta L \log\left(\mathcal{N}(\mathcal{H},\epsilon)\right)\epsilon}{n}}&
    \leq \frac{32\beta L \log\left(\mathcal{N}(\mathcal{H},\epsilon)\right)\epsilon}{n\eta}+\frac{\eta }{128}\|\hat{f}_i-f_\ast\|^2\\
    \frac{\beta \log\left(\mathcal{N}(\mathcal{H},\epsilon)\right)\epsilon \|\hat{f}_i-f_\ast\|}{n}
    &\leq \frac{32\beta \log\left(\mathcal{N}(\mathcal{H},\epsilon)\right)^2\epsilon^2}{n^2\eta}+\frac{\eta }{128}\|\hat{f}_i-f_\ast\|^2.
  %\end{aligned}
  \end{align*}
 Substituting the above  inequation into \eqref{equation-first-equation}, we can obtain that
 \begin{align*}
   &~~~~R(\hat{f}_i)-R(f_\ast)+\frac{\eta}{4}\|\hat{f}_i-f_\ast\|^2\\
   &\leq \frac{\beta \log\left(\mathcal{N}(\mathcal{H},\epsilon)\right)\|\hat{f}_i-f_\ast\|^2}{n}+
   \frac{2\beta \log\left(\mathcal{N}(\mathcal{H},\epsilon)\right)(R(\hat{f}_i)-R(f_\ast))}{n\eta}+\frac{8\beta \log(2/\delta)}{n^2\eta}\\
   &~~~+\frac{64\beta R_\ast\log(2/\delta)}{n\eta}+\frac{32\beta^2\epsilon^2}{\eta}+
   \frac{32\beta L \log\left(\mathcal{N}(\mathcal{H},\epsilon)\right)\epsilon}{n\eta}+
   \frac{32\beta \log^2\left(\mathcal{N}(\mathcal{H},\epsilon)\right)\epsilon^2}{n^2\eta}\\
   &\overset{\eqref{equation-12}}{\leq}
   \frac{\eta}{4}\|\hat{f}_i-f_\ast\|^2+\frac{1}{2}(R(\hat{f}_i)-R(f_\ast))+\frac{8\beta \log(2/\delta)}{n^2\eta}\\
   &~~~+\frac{64\beta R_\ast\log(2/\delta)}{n\eta}+\frac{32\beta^2\epsilon^2}{\eta}+
   \frac{32\beta L \log\left(\mathcal{N}(\mathcal{H},\epsilon)\right)\epsilon}{n\eta}+
   \frac{32\beta \log^2\left(\mathcal{N}(\mathcal{H},\epsilon)\right)\epsilon^2}{n^2\eta}.
 \end{align*}
 Thus, with $1-2\delta$, we have
 \begin{align}
    \label{equation-14}
    \begin{aligned}
    R(\hat{f}_i)-R(f_\ast)&\leq
    \frac{16\beta \log(2/\delta)}{n^2\eta}+\frac{128\beta R_\ast\log(2/\delta)}{n\eta}+\frac{32\beta^2\epsilon^2}{\eta}\\
    &~~~~+
   \frac{64\beta L \log\left(\mathcal{N}(\mathcal{H},\epsilon)\right)\epsilon}{n\eta}+
   \frac{64\beta \log^2\left(\mathcal{N}(\mathcal{H},\epsilon)\right)\epsilon^2}{n^2\eta}.
   \end{aligned}
  \end{align}
  Combining \eqref{equaiton-strongly-ff} and \eqref{equation-14},
  with $1-2\delta$,
  we have
  \begin{align*}
    R(\bar{f})-R(f_\ast)&\leq
    \frac{16\beta \log(2m/\delta)}{n^2\eta}+\frac{128\beta R_\ast\log(2m/\delta)}{n\eta}+\frac{32\beta^2\epsilon^2}{\eta}+
   \frac{64\beta L \log\left(\mathcal{N}(\mathcal{H},\epsilon)\right)\epsilon}{n\eta}+\\
   &~~~~+
   \frac{64\beta \log^2\left(\mathcal{N}(\mathcal{H},\epsilon)\right)\epsilon^2}{n^2\eta}
  -\frac{\eta}{4m^2}\sum_{i,j=1,i\not=j}^m\|\hat{f}_i-\hat{f}_j\|^2.
  \end{align*}
\end{proof}

%\begin{theorem}
%  At least $1-2\delta$,
%  if
%  \begin{align}
%    \label{equation-12}
%    n\geq \frac{4\beta\log\left(\mathcal{N}(\mathcal{H},\epsilon)\right)}{\eta},
%  \end{align}
%  we have
%  \begin{align}
%    \label{equation-13}
%    \begin{aligned}
%    R(\hat{f}_i)-R(f_\ast)&\leq
%    \frac{16\beta \log(2/\delta)}{n^2\eta}+\frac{128\beta R_\ast\log(2/\delta)}{n\eta}+\frac{32\beta^2\epsilon^2}{\eta}\\
%    &~~~~+
%   \frac{64\beta L \log\left(\mathcal{N}(\mathcal{H},\epsilon)\right)\epsilon}{n\eta}+
%   \frac{64\beta \log^2\left(\mathcal{N}(\mathcal{H},\epsilon)\right)\epsilon^2}{n^2\eta}
%   \end{aligned}
%  \end{align}
%\end{theorem}
\bibliographystyle{abbrv}
\bibliography{NIPS2018}
\end{document}
